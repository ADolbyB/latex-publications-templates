\documentclass[11pt]{article} % Can Also be 'book' or 'report'
\usepackage[margin=1in]{geometry} % Set Margins to 1 inch standard
\usepackage{amsmath, amssymb, amsfonts}
\usepackage[table]{xcolor} % Used for shaded tables. loads also »colortbl«
\pagestyle{empty} % Turn Off Page Numbering
\parindent 0px % Turn Off Paragraph Indentation

\title{\LaTeX\ Tutorial 3: Brackets, Tables and Arrays}
\author{Joel M. Brigida: ADolbyB}

\begin{document}
\maketitle % Document Title

\section{Brackets:}
\thispagestyle{empty} % Remove Stubborn 1st Page Number

The distributive property states that $a(b+c) = ab + ac$ for all $a, b, c \in \mathbb{R}$.\\[6pt]
The equvalence class of $a$ is $[a]$.\\[6pt]
The set $A$ is defined to be $\{1, 2, 3\}$.\\[6pt]
The ticket costs $\$15.45$\\[6pt]
This Looks Bad: $$ 2(\frac{1}{x^2 +1}) $$
Scaling Parenthesis Larger For Proper Format: $$ 2\left(\frac{1}{x^2+1}\right) $$
Other Examples:\\[6pt]
Square Brackets:
$$ 2\left[\frac{1}{x^2+1}\right] $$
Curly Braces:
$$ 2\left\{\frac{1}{x^2+1}\right\} $$
Angle (Vector) Brackets:
$$ 2\left\langle\frac{1}{x^2+1}\right\rangle $$
Absolute Value:
$$ 2\left|\frac{1}{x^2+1}\right| $$
Evaluation:
$$ \left.\frac{\mathrm{d}y}{\mathrm{d}x}\right|_{x=1}^{x=5} $$
Complex Fractions:
$$ \left(\frac{1}{1 + \left(\frac{1}{1+x}\right)}\right) $$

\pagebreak

\section{Tables:}
Notice this doesn't look right:\\[6pt]
\begin{tabular}{|c||c|c|c|c|c|} % 6 center aligned column: r for right aligned, l for left aligned
    \hline
    $x$ & 1 & 2 & 3 & 4 & 5 \\ \hline
    $f(x)$ & $\frac{1}{2}$ & 11 & 12 & 13 & 14 \\ \hline
\end{tabular}

\vspace{1cm} % 1 cm vertical space

\begin{table}[h!] % [h!] I want it HERE!
\centering
\def\arraystretch{1.4}
\caption{Values For $f(x)$}
\begin{tabular}{|c||c|c|c|c|c|}
    \hline
    $x$ & 1 & 2 & 3 & 4 & 5 \\ \hline
    $f(x)$ & $\frac{1}{2}$ & 11 & 12 & 13 & 14 \\ \hline
\end{tabular}
\end{table}

\vspace{1cm} % 1 cm vertical space

\begin{table}[h!] % [h!] I want it HERE!
\centering
\def\arraystretch{1.4}
\caption{Values For $f(x)$}
\begin{tabular}{|c|c|}
    \hline
    $f(x)$ & $f'(x)$ \\ \hline
    $x > 0$ & The Function $f(x)$ is increasing. \\ \hline
\end{tabular}
\end{table}

\vspace{1cm} % 1 cm vertical space

\begin{table}[h!] % [h!] I want it HERE!
\centering
\caption{Description of $f(x)$}
\def\arraystretch{1.4}
\begin{tabular}{|l|p{4in}|} % p for paragraph
    \hline
    $f(x)$ & $f'(x)$ \\ \hline
    $x > 0$ & The Function $f(x)$ is increasing as long as the function does not cross the $y=0$ boundary,
    in which case it becomes undefined. This happens once at the value of $x = 3\pi$. \\ \hline
\end{tabular}
\end{table}

\vspace{1cm} % 1 cm vertical space

\begin{table}[h!]
\centering
\rowcolors{2}{gray!25}{white}
\def\arraystretch{1.4}
\caption{Example Derivatives}
\begin{tabular}{|c|c|}
\rowcolor{gray!50}
    \hline
    $f(x)$ & $f'(x)$ \\ \hline
    $x^{2}$ & $2x$\\ \hline
    $x^3$ & $3x^2$ \\ \hline
    $x^4$ & $4x^3$ \\ \hline
    $x^5 + 4x^4$ & $5x^4 + 16x^3$ \\ \hline
\end{tabular}
\end{table}

\pagebreak

\section{Arrays:}

\begin{align} % Everything in align is in math mode.
    5x^2 + 13x + 3 \, \, \text{Example text in math mode: place some text here.}
\end{align}
    Here are aligned numbered equations:
\begin{align} % Everything in align is in math mode.
    5x^2 + 13x + 3 &= 12x + 4\\
    5x^2 - 9 &= x + 3\\ % &= Lines up equal signs
    15x^3 + 14x^2 - 3x + 3 &= 8x^2
\end{align}
    Here are aligned non-numbered equations:
\begin{align*} % Asterisk removes numbering
    5x^2 + 13x + 3 &= 12x + 4\\
    5x^2 - 9 &= x + 3\\ % &= Lines up equal signs
    15x^3 + 14x^2 - 3x + 3 &= 8x^2
\end{align*}

\end{document}