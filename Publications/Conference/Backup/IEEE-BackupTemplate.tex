\documentclass[conference]{IEEEtran}
\IEEEoverridecommandlockouts
% The preceding line is only needed to identify funding in the first footnote. If that is unneeded, please comment it out.
\usepackage{cite}
\usepackage{amsmath,amssymb,amsfonts}
\usepackage{algorithmic}
\usepackage{graphicx}
\usepackage{textcomp}
\usepackage{xcolor}
\def\BibTeX{{\rm B\kern-.05em{\sc i\kern-.025em b}\kern-.08em
    T\kern-.1667em\lower.7ex\hbox{E}\kern-.125emX}}

\begin{document}

\title{From Crisis to Control: Sensor Networks and Smart Systems Shaping Effective Disaster Management \\
	\thanks{Special thanks to all our families for their support through our journey.}
}

\author{
	\IEEEauthorblockN{
		Jehad Ismail\IEEEauthorrefmark{1},
		Joel Brigida\IEEEauthorrefmark{2},
		Bartholdy Alexandre\IEEEauthorrefmark{3},
		Stephanie Val\IEEEauthorrefmark{4} and
		Isabela Costa\IEEEauthorrefmark{5}}
	\IEEEauthorblockA{
		\IEEEauthorrefmark{1}
		College of Engineering and Computer Science\\
		Florida Atlantic University, Boca Raton, FL USA 33431\\
		Email: aismail2021@fau.edu}
	\IEEEauthorblockA{
		\IEEEauthorrefmark{2}
		Design and Prototyping\\
		Dark Ridge LLC, West Palm Beach, FL 33415\\
		Email: joel@joelbrigida.com}
	\IEEEauthorblockA{
		\IEEEauthorrefmark{3}
		College of Engineering and Computer Science\\
		Florida Atlantic University, Boca Raton, FL USA 33431\\
		Email: balexandre2019@fau.edu}
	\IEEEauthorblockA{
		\IEEEauthorrefmark{4}
		College of Engineering and Computer Science\\
		Florida Atlantic University, Boca Raton, FL USA 33431\\
		Email: sval2021@fau.edu}
	\IEEEauthorblockA{
		\IEEEauthorrefmark{5}
		College of Engineering and Computer Science\\
		Florida Atlantic University, Boca Raton, FL USA 33431\\
		Email: icosta2021@fau.edu}
}

\maketitle

\begin{abstract}
	This paper explores the role of sensor networks and smart systems in effective disaster management. 
	It discusses their contributions to prevention, preparation, response, and recovery phases, highlighting 
	their transformative impact. The implementation of wireless sensor networks for landslide monitoring is 
	presented as a case study. The paper also touches upon the use of sensor networks in various natural 
	disasters and emphasizes the importance of emergency communication, remote monitoring through drones 
	and cameras, and resilient infrastructure. Overall, it showcases the vital role of sensor networks and 
	smart systems in proactive disaster management.\par
\end{abstract}

\begin{IEEEkeywords}
	Smart Systems, Sensor Networks, IoT, Embedded Systems\par
\end{IEEEkeywords}

\section{Introduction} % Section I
Sensor networks and smart systems play a crucial role in our modern world, offering a multitude of benefits
to society, with a primary focus on enhancing safety. Particularly in the field of disaster management, these
technologies have emerged as indispensable tools. The utilization of specialized sensors designed to detect
and respond to natural disasters, coupled with the implementation of alert systems, decision-making support,
remote monitoring capabilities, and resilient infrastructure, has revolutionized the landscape of effective
disaster management. Sensor networks and smart systems have truly transformed the way we approach and mitigate
the impact of catastrophes, ensuring a more proactive and resilient response.\par

\section{Sensor Networks} % Section II

Sensor networks and smart systems in disaster management serve four primary objectives: prevention, preparation,
response, and recovery. These interconnected technologies play a pivotal role in each phase of disaster
management, contributing to a more comprehensive and
effective approach.\par

In reference to prevention, sensor networks and smart systems work proactively to prevent disasters or reduce
theird impact. By constantly monitoring environmental conditions such as the temperature, humidity, air quality,
or seismic activity, sensor networks can allow for early warning signs and trigger alerts. With regard to
preparation, sensor networks and smart systems assist in the preparation of disaster management. They can
deliver real-time data and analytics that allow authorities to measure risks, plan response approaches, and
allocate resources efficiently. In terms of response efforts, sensor networks and smart systems play a crucial
role in response operations. They provide real-time data on environmental conditions, infrastructure integrity,
and the condition of the affected populations. This information will help emergency responders to make
informed decisions, prioritize actions, and allocate resources to areas where they are needed the most.
As for recovery, sensor networks and smart systems continue to assist in recovery efforts. They provide
crucial data for damage assessment, structural health monitoring, and environmental recovery. Smart systems
facilitate data analytics and decision support tools that assist in resource allocation, rebuilding
strategies, and long-term recovery plans.\par

The utilization of sensor networks in monitoring natural disasters has garnered significant attention from
researchers and engineers due to its widespread applicability. An intriguing case in point is the implementation
of wireless sensor networks specifically designed for landslide monitoring. Kotta et al. \cite{b1} offer a
compelling solution in the form of a wireless sensor network system that relies on accelerometers to detect
vibrations associated with landslides. Through their experiments, they observed that when the accelerometer's
value surpassed 1 gravity, it served as a critical indicator of substantial mass sliding and hazardous
conditions.\par

This innovative wireless sensor network system demonstrates how technology can effectively contribute to
disaster management. By employing accelerometers as sensing devices, the system can detect even minute
changes in vibration levels, allowing for the early identification of potential landslides. The obtained
data provides valuable insights into the intensity of mass sliding, enabling authorities to assess the
severity of the situation and take appropriate measures to safeguard lives and property.\par

The study conducted by Kotta et al. \cite{b1} showcases the immense potential of wireless sensor networks
in mitigating the risks associated with landslides. The implementation of such advanced monitoring systems
not only enhances the accuracy of landslide detection but also improves the overall response time and
decision-making during critical situations. Consequently, these findings pave the way for the development
of more robust and efficient sensor network solutions that aid in the proactive management of natural
disasters.\par

There are several types of natural disasters and each disaster uses sensor networks in its own unique ways.
The use of sensor networks for landslides is solely just one example. With earthquakes, sensor networks are
deployed to earthquake-prone areas to detect seismic activity and monitor the ground motion. Seismometers,
accelerometers, and geophones are some of the sensors found in these networks that measure the intensity,
duration, and frequency of the ground shaking. For floods, it`s pretty simple: water level sensors are placed
in rivers, streams, and flood-prone areas and they continuously measure the water levels, predict flood events
and issue warnings in real-time. With wildfires, there are smoke detectors and infrared sensors within the
sensor networks that detect smoke and abnormal rise in temperature which can trigger alerts. Other technologies
used include thermal cameras and remote sensors to monitor the behavior of fire, heat signatures, and patterns
of the spreading fire. Pertaining to hurricanes and storms, weather monitoring stations use anemometers,
barometers, and rain gauges within the sensor networks. These are responsible for measuring wind speed,
atmospheric pressure, and precipitation. As for volcanoes, sensor networks are deployed near active volcanoes
to monitor their volcanic activity. The types of sensors used here are seismic sensors and gas detectors
and they are used to track ground vibrations, gas emissions, and any changes in volcanic activity. Collecting
all this data allows for clear insights into volcanic eruptions. Lastly, for tsunamis Buoy-based sensors
are installed in coastal waters to detect changes in sea level and to transmit real-time data. The Buoy-based
sensors are integrated with seismometers that can detect underwater earthquakes which are associated with tsunamis.
With these sensor networks in place, early warning systems can initiate evacuation procedures.\par

In general, sensor networks are instrumental in effective disaster management. They enable prevention through 
early warning systems, support preparation efforts by providing real-time data for risk assessment, facilitate 
response operations by offering critical information to emergency responders, and aid in the recovery phase by 
assessing damage and monitoring environmental conditions. With their ability to continuously monitor and collect 
data, sensor networks empower authorities to make informed decisions and take proactive measures, ultimately 
saving lives and minimizing the impact of disasters. As technology advances, sensor networks will continue to 
evolve, enhancing their role in disaster management and contributing to safer and more resilient communities.\par

\section{Emergency Communication And Alert Systems} % Section III

Lorem ipsum dolor sit amet, consectetuer adipiscing elit, sed diam nonummy nibh euismod tincidunt ut laoreet

\section{Remote Monitoring} % Section  IV
Remote Monitoring in disaster management involves using equipment such as cameras, drones, and the like, 
to survey affected areas, sensing temperatures, relative humidity levels, leaks, ventilation, dew points, 
adverse weather developments, and so on; Equipment that also relays ``information regarding power outages 
and weather changes ... [and] [t]he size of the equipment and components vary by manufacturer and model 
\cite{b4}.'' This equipment of sensors plays a focal role in the monitoring and alerting of disaster 
management, contributing to a more comprehensive approach in response to these disasters. With remote 
monitoring systems, recovery specialists and first responders can set considerations to maintain ideal 
circumstances in recovery efforts contributing to disaster readiness.\par

One example showing this is the application of drones, where drones have become an effective part of harm
reduction, seeing that their use ``has rapidly evolved over the past decade ... [in] a variety of fields ... and [has] becom[e] increasingly used in disaster management or humanitarian aid \cite{b5}.'' More specifically, the application of drones in disaster control has expanded to search and rescues in that they can ``reduce the time required to locate victims and the time required for subsequent intervention by searching a large area in a short period of time, ... providing critical information to rescuers about the route that needs to be taken. Additionally, ... searching for alive victims buried beneath rubble using sensors such as noise sensing, binary sensing, vibration, and heat sensing \cite{b5}.'' Thus, when it comes to disaster management, drones have significant potential in helping in searching for lost and trapped civilians due to disasters such as cave-ins, floods, hurricanes, and the like. Remote monitoring, with drones, has the potential to make the lives of people, especially in high-risk areas, feel more at ease and safer so that their chances of survival are higher.\par

Another example of the potential benefits of remote monitoring in disaster management is found in landslide
monitoring. Like in the use of drones, the Internet of Things (IoT) ``plays a major role for the purpose of
monitoring natural disasters \cite{b7}'', specifically in things like landslides; Seeing that landslides ``cause
more than \$100 million in direct damage and cause thousands of fatalities \cite{b7}'' and so, in ``order to
mitigate the landslide hazard, several landslide monitoring techniques have been developed over the last
decades \cite{b7}''.  Remote sensing, a commonly used technique, is one of these methods, which is ``mainly
used in landslide detection, fast characterization, and mapping applications \cite{b7}'' to ``gather
information about the distribution and kinematics of surface displacements. Remote sensing makes use of
aircraft, spacecraft, or terrestrial-based platforms \cite{b7}.'' Even though the manual live supervision
is limited due to the revisit period of satellites, it is typically ``suitable for mapping vulnerable
areas...[and] monitoring displacements over a large area...with 3-D capabilities \cite{b7}”. To elaborate
further, these remote techniques collect data about an area using satellites, airborne, or ground-based
sensors. ``[The] most commonly used techniques are based on laser, radar, and infrared sensor[s] \cite{b7}.''
For example, these remote sensors, specifically terrestrial laser scanning (TLS), have been used for active
landslides in the French Alps. Using automated monitoring to track rockfalls and landslides, providing near
real-time monitoring/change detection for data collection. This automated monitoring helps find and predict
areas at high risk of being affected and buried, allowing at-risk people to be alerted and relocated when
needed or for businesses to take the proper precautions.\par

An additional example of the potential benefits of remote monitoring in disaster management is also seen in Water
Level Monitoring/Flood monitoring. Where the Internet of Things (IoT) ``plays a major role in the purpose of
monitoring natural disasters \cite{b7};'' like river behavior, and how it may ``help mitigate or prevent future
disasters \cite{b6}.'' With that said, seeing that floods are amongst the ``most common and devastating of all
natural hazards [accounting for 41\% of all natural perils that occurred globally in the last decade] \cite{b3}''.
Remote monitoring, correspondingly, has the potential to curb flood-related deaths and the cost of damages.
And so, to mitigate the hazards of floods caused by things like water levels and storms, ``activities exploring
how camera images and wireless sensor data...[that] can improve flood management \cite{b3}'' have been
developed and utilized. One method being computer-vision, a commonly used technique based on cameras and
``relevant images from existing urban surveillance cameras [that] are captured and processed to improve
decision-making \cite{b3}.'' These remote camera-based systems are more capable and commercial in that
they ``involve low equipment cost and wide aerial coverage..., enabling the detection of flood levels at
multiple points \cite{b3}.'' In other words, they have the advantage over traditional fixed-point methods
of sensing in that they have an extensive reportage built on ``image processing techniques that have been
widely applied in many fields, including aerospace, medicine, traffic monitoring, and environmental object
analysis \cite{b3}.'' An aspect of how computer vision helps with disaster management of floods is how
it aids with monitoring water levels in places such as lakes, rivers, and other potentially disastrous
water sources since they are of extreme importance when it comes to early warning signs of a flood.
Computer vision is useful in monitoring water levels with things like ``Image filtration...[which] plays
a vital role in estimating water levels \cite{b3}.'' To elaborate further, a difference (image) method is
utilized to analyze images of ``the region of interest (ROI) between the previous and current frame and
then outputting a level of water...the water level is then estimated from the y-axis of the edged image
\cite{b3}.'' This remote solution of difference has been utilized a few times and has shown to be dependable
with adequate accuracy. And so, computer vision has been of great help and importance when it comes to disaster
management and monitoring disasters caused by flooding with computers and cameras and has the budding to
advance ``flood inundation mapping, debris flow estimation, and post-flood damage estimation[s]
\cite{b3}.''\par

Overall, those are just some ways remote monitoring helps in disaster management of things like search and rescue,
floods, and landslides. With the use of things like; drones, remote sensing, and computer vision, the equipment
makes preventing the horrid outcomes of such disasters much easier.\par

\section{Resilient Infrastructure} % Section V
Resilient Infrastructure incorporates more than just device security. Its very existence should not pose a threat
to your safety. We buy and drive cars and trust that at 70 MPH on the highway, the wheels won't fall off, or the
steering wheel suddenly doesn't work. The vast new world of IoT devices must have ``SOMETHING'' so that they are
trusted by the public. This can include everything down to where the minerals to make the batteries supplied with
the device are mined and processed. \par

Data Resilience \cite{b8} needs to be discussed. Lose or corrupt the data that is transferred across the network, your IoT network is basically junk.\par

Resilient Infrastructure - How IoT can contribute to building resilient infrastructures that can withstand and recover from disasters. Infrastructures equipped with IoT sensors to trigger early warning systems\par

In an effort to help constantly improve IoT Device Security....

\par

\par

\par

\par

\par

\section{Conclusion} % Section VI
% NEED AN ACTUAL CONCLUSION
Here we have a conclusion. Remember that in a conclusion, there is no new information presented, and all it 
does is sum up your observations, to which you declare a solution and briefly defend it.
% NEED AN ACTUAL CONCLUSION

\bibliographystyle{./IEEEtran}
\bibliography{./IEEEabrv, ./COT4930}


% % Bibliography Entries are all at the end in IEEE Template...
\begin{thebibliography}{00}
	\bibitem{b1} Herry Z Kotta, Kalvein Rantelobo, Silvester Tena, and Gregorius Klau, ``Wireless sensor
	net-work for landslide monitoring in nusa tenggara timur,'' TELKOMNIKA, (TelecommunicationComputing
	Electronics and Control), 9(1):9-18, 2011
	\bibitem{b2} D. Prasad, A. Hassan, D. K. Verma, P. Sarangi and S. Singh, ``Disaster Management System
	Using Wireless Sensor Network: A Review,'' 2021 International Conference on Computational Intelligence
	and Computing Applications (ICCICA), Nagpur, India, 2021, pp. 1-6, doi: https://doi.org/10.1109/
	ICCICA52458.2021.9697236.
	\bibitem{b3} Arshad, Bilal, et al. ``Computer Vision and IoT-Based Sensors in Flood Monitoring and
	Mapping: A Systematic Review.'' Sensors, vol. 19, no. 22, 16 Nov. 2019, p. 5012, https://doi.org/
	10.3390/s19225012. Accessed 30 Sept. 2020.
	\bibitem{b4} ``How Disaster Recovery Teams Use Remote Monitoring.'' Www.polygongroup.com,
	www.polygongroup.com/en-US/blog/how-remote-monitoring-services-assist-disaster-recovery-teams/.
	Accessed 30 June 2023.
	\bibitem{b5} Mohd Daud, Sharifah Mastura Syed, et al. ``Applications of Drone in Disaster Management: A
	Scoping Review.'' Science and Justice, vol. 62, no. 1, 1 Jan. 2022, pp. 30-42, www.sciencedirect.com/
	science/article/pii/S1355030621001477, https://doi.org/10.1016/j.scijus.2021.11.002.
	\bibitem{b6} Moreno, Carlos, et al. ``RiverCore: IoT Device for River Water Level Monitoring over
	Cellular Communications.'' Sensors, vol. 19, no. 1, 2 Jan. 2019, p. 127, www.ncbi.nlm.nih.gov/pmc/
	articles/PMC6338933/, https://doi.org/10.3390/s19010127.
	\bibitem{b7} Thirugnanam, Hemalatha, et al. ``Review of Landslide Monitoring Techniques with IoT
	Integration Opportunities.'' IEEE Journal of Selected Topics in Applied Earth Observations and Remote
	Sensing, vol. 15, 2022, pp. 5317-5338, https://doi.org/10.1109/jstars.2022.3183684. Accessed 31 July
	2022.
	\bibitem{b8} K., Vitaly. ``Data Resilience in Large-Scale IOT Deployments.'' Forbes, 21 Apr. 2022, www.forbes.com/sites/forbestechcouncil/2021/11/19/data-resilience-in-large-scale-iot-deployments/?sh=1a1f7362c88c. 
	\bibitem{b9}
	\bibitem{b10}

% 	%% Free IEEE Website Citation Generator: https://www.mybib.com/tools/ieee-citation-generator
	
% 	%% Author initials. Last name. “Page title.” Website Name. URL (accessed Month Day, Year).

% 	%% Example IEEE Reference:

% 	%% P. Bhandari. “Nominal data | Definition, examples, data collection & analysis.” Scribbr.  https://www.scribbr.com/statistics/nominal-data/ (accessed Aug. 11, 2022).


\end{thebibliography}

\vspace{12pt}
\end{document}