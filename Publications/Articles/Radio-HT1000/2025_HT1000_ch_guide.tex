\documentclass[12pt,letterpaper]{article}
\usepackage[margin=0.75in]{geometry}    % 3/4 inch margins
\usepackage{amsfonts, amssymb, amsmath} % all the fun math stuff
\usepackage{graphicx}                   % For images
\usepackage{hyperref}                   % For hyperlinks
\usepackage{caption}                    % Better captions
\usepackage{indentfirst}                % Force Sections 1st paragraph to Indent like the rest.
\usepackage{float}                      % [H] to put images 'HERE'
\usepackage{enumerate}                  % Customize Enumerated Lists
\usepackage{url}                        % used for custom \url{....} methods
\usepackage{enumitem}                   % used to break enumerated lists, insert text, and resume enumeration.
\hypersetup{%
    colorlinks=true,                    % colored text rather than box outlined links (cleaner looking)
    urlcolor=blue
}
\let\oldurl\url                         % Makes \url{} automatically use a custom size.
\renewcommand{\url}[1]{{\small\oldurl{#1}}} % custom \small \url size

% ======================= Document Start ===================================
\begin{document}
% ====================== Title Page Start ==================================
\title{A General Handheld Radio Primer}
\author{Joel M. Brigida}
\date{}                                 % leave blank for no date or \today for current date.
\thispagestyle{empty}                   % No page number on title page

\begin{titlepage}                       % Custom titlepage instead of the default \maketitle.
    \centering
    \vspace*{1cm}
    \rule{\textwidth}{1pt}              % Top title rule (bar)

    \vspace{.7\baselineskip}            % Title
    {\huge \textbf{A General Handheld Radio Primer \\ \vspace*{.5cm}}}
    \LARGE Using Surplus HT1000s in the MURS Band % Subtitle
    
    \rule{\textwidth}{1pt}              % Bottom title rule (bar)
    \vspace{1cm}
    
    \large Written in \LaTeX\           % Set this size for the remaining titlepage.
    
    \vspace{3cm}                        % More authors can be inserted here with additional minipages.
    \vfill

    Joel Brigida, BSCE \\               % University and date information at bottom of titlepage.
    R \& D, Engineering. \\
    % src: https://www.overleaf.com/latex/templates/sample-report-title-page-for-an-university-assignment-or-project/qwqpmbbmfygn
    \begin{minipage}{.5\textwidth}
        \centering
        {\normalsize \texttt{\href{https://github.com/ADolbyB/}{GitHub}}} \, 
        {\normalsize \texttt{\href{https://www.linkedin.com/in/joelmbrigida/}{LinkedIn}}}
    \end{minipage}

\end{titlepage}

\medskip
\pagenumbering{arabic}                  % 1,2,3 type page numbering.
\setcounter{page}{1}                    % Forces THIS page to be page 1

% === ToC Page  1 ==========================================================
\renewcommand{\contentsname}{Table of Contents} % Custom name for Table of Contents
\begin{center}
    \item \tableofcontents                    % Table of Contents appears on page 1
\end{center}

\clearpage
% === Page 2 ===============================================================
\begin{center}
    \item \section{Disclaimer}
\end{center}

Please note that none of the information in this manual is meant to be legal advise in any way. The author of
this document does not assume any responsibility for the accuracy of the information in this document. This 
document should be regarded as the author's own opinion, and nothing more for legal reasons even though this is a 
well researched subject in the author's own expertise.

\clearpage                              % Start main content on page 2
% === Page 3 ===============================================================

\begin{center}
    \item \section{General Information}
\end{center}

This is a general handheld radio primer that can be adapted to any radio \textit{band plan}.  
This plan happens to describe operation on the \textit{MURS} portion of the \textit{FCC Part 95} frequency
allocations, using surplus Motorola HT1000s, but can be adapted to any radio with any channel configuration, 
since the principles of operation are all the same. This manual was originally written for a client of the
author's located in SE Massachusetts, and is now being modified for release to this public repository.

\medskip
\noindent In this guide, the following nomenclature will be used:

\medskip
\begin{itemize}
    \item \textit{Italics} are used for keywords or vocabulary that is important to learn.
    \item \textsc{Small Caps} refers to a unique work employment position or title.
    \item \textbf{Bold Face} is used for emphasis.
    \item The ``$\approx$'' symbol is used, precedes a number and means ``approximately''.
    \item The abbreviation ``Hz'' for Hertz is used, which is a measurement of frequency. Hz means ``cycles per 
        second,'' which is to say how many times one radio wave cycle or period repeats itself per second. Also note:
        \begin{itemize}
            \item kHz = kilohertz: thousands of cycles per second.
            \item MHz = megahertz: millions of cycles per second.
        \end{itemize}
\end{itemize}

\medskip
This guide focuses on some slightly advanced topics that the client was interested in, such as some theory 
and some higher level operating procedures. Since the client had never had any radio experience before, the purpose 
of this guide was a reinforcement of rules, regulations, types of operation, and what to do in case of a problem in 
the field, as well as the author's view on what happens when the radios need service, etc. This in particular 
helped my client be more confident in their own operating procedure, as well as provide a level of confidence in 
the author that I am able to troubleshoot and repair any problems that may arise over time. This guide should also 
help explain various radio functions and help explain some useful, often overlooked features of the Motorola 
HT1000s. The actual manual that is screenshot on page 6 is a free downloadable PDF that is available on 
\textit{The Internet Archive}. There is a link to this manual as well as a link to the Radio Service 
Software User's Manual placed in the ``Links'' section of this document. This guide is provided as a small bridge
between the knowledge gap that exists between newer radio users and the Motorola manuals.

\clearpage
% === Page 4 ========================================================================
\begin{center}
    \item \subsection{MURS Frequency Allocation and Related Info}
\end{center}

These particular HT1000 radios operate in the MURS band in the \textit{VHF-Hi} portion of 
\textit{The Electromagnetic Spectrum}. MURS stands for ``Multi Use Radio Service'' and operates under the 
FCC Part 95 rules, which in this particular case is unlicensed spectrum, meaning no license is required to 
use it. This is unlike Amateur (ham) Radio, which operates under \textit{FCC Part 97} Rules, and has 5 
different license classes, 2 of which are \textit{legacy} only licenses.

\medskip
What is VHF-Hi Band? VHF is ``Very High Frequency'', which is considered to be
$30 - 300\text{ MHz}$ in frequency or $10\text{ meters} - 1\text{ meter}$ in wavelength. The
Hi or Lo designation splits the entire VHF band into 2 parts. This split happens at 
$\approx$ 100 MHz. So if your radio (or television) operates above  $\approx$ 100 MHz but 
below 300 MHz, then it operates in the VHF-Hi band. As a comparison, the FM Broadcast radio in automobiles
receives from 88 MHz to 108 MHz, so it operates at the upper edge of VHF-Lo band, and crosses into the 
lower edge of the VHF-Hi band. \textit{FCC Part 73} encompasses the rules for all broadcast services, including 
television and radio.

\medskip
For this particular client in SE Massachusetts, the VHF MURS band was chosen for these particular reasons:
\medskip
\begin{enumerate}
    \item MURS operation is license free, unlike other services in the VHF-Hi band.
    \item The client's location is rural with plenty of foliage. VHF-Hi band has superior coverage in this
        type of environment over higher frequency bands.
    \item VHF-Hi band is an excellent compromise between antenna length and terrain penetration, unlike lower
        frequency bands which require much larger antennas.
\end{enumerate}

\medskip
Something else to mention is that these HT1000 radios are \textit{FCC Part 90} certified for use in the
VHF-Hi band, which in the case of this bandsplit of Motorola HT1000s means they can operate from 
136 MHz to 174 MHz. Why? Because that is the allocated spectrum for anything operating 
under FCC Part 90 rules in the VHF-Hi band. Lower than 136 MHz is used for aircraft (FCC Part 87), and above
174 MHz is used for broadcast television (FCC Part 73). 

\medskip Also note why we are allowed to use an FCC Part 90 radio in the MURS band: the two sets of rules 
(Part 90 and Part 95) have overlap and the Part 90 radio type certification can meet or exceed any rules in the
FCC Part 95 regulations, given that we can operate within the FCC's Part 95 frequency, bandwidth and power 
allocations.

\clearpage
% === Page 5 ========================================================================
\medskip
So what are the MURS frequency allocations? These \textit{channel allocations} are numbered and named by 
the FCC as follows under Part 95:

\medskip
\begin{enumerate}
    \item Name: MURS 1.
    \begin{itemize}
        \item Freq: 151.82 MHz.
        \item Max \textit{Bandwidth}: 11.25 MHz.
        \item Max Power: 2 Watts \textit{ERP}.
    \end{itemize}
    \item Name: MURS 2.
    \begin{itemize}
        \item Freq: 151.88 MHz.
        \item Max \textit{Bandwidth}: 11.25 MHz.
        \item Max Power: 2 Watts \textit{ERP}.
    \end{itemize}
    \item Name: MURS 3
    \begin{itemize}
        \item Freq: 151.94 MHz.
        \item Max \textit{Bandwidth}: 11.25 MHz.
        \item Max Power: 2 Watts \textit{ERP}.
    \end{itemize}
    \item Name: MURS 4 aka Blue Dot.
    \begin{itemize}
        \item Freq: 154.57 MHz.
        \item Max \textit{Bandwidth}: 20 MHz.
        \item Max Power: 2 Watts \textit{ERP}.
    \end{itemize}
    \item Name: MURS 5 aka Green Dot.
    \begin{itemize}
        \item Freq: 154.60 MHz.
        \item Max \textit{Bandwidth}: 20 MHz.
        \item Max Power: 2 Watts \textit{ERP}.
    \end{itemize}
\end{enumerate}

This is a list the FCC issues for MURS channel allocation and acts as the guide to proper legal operation. These 
are the specifications, or ``rules'' that the FCC provides to the public so the MURS channels are used legally and 
do not cause \textit{interference} on adjacent channels or some other service that is not MURS. These FCC rules 
are also good for \textit{interoperability} so that everyone wanting to use the MURS band uses the same 
settings, the audio received from the radio sounds correct, and all the radios are on the same frequencies.

\clearpage
% === Page 6 ========================================================================
\begin{center}
    \item \subsection{The Motorola \textit{Jedi Series} Handhelds}
\end{center}

% TODO: Fix the figure placed "here!" that overlaps the text. FIXED JB.
\begin{figure}[h!] % [h!] for here! to place figure on 1st page. Otherwise its on 2nd Page.
    \center
    \includegraphics[width=\linewidth]{Jedi_Series.png}
    \caption{The HT1000 is a part of the \textit{Jedi Series} of handheld transceivers.}
\end{figure}

\clearpage
% === Page 7 ========================================================================
The above linked guide is a free public download, but is a 170 page document. It is the factory service manual that an 
\textsc{RF Technician} or \textsc{RF Engineer} could refer to for service and \\ maintenance of the Jedi Generation of 
handheld radios. It has all the part numbers and pictures for every piece the radio contains. The service manual 
also contains other important information just like this author used in his job as \textsc{Production Engineer} at 
an engineering firm in South Florida. Most of the information needed for this job was:

\begin{itemize}
    \item Printed Circuit Board (PCB) Physical Layout.
    \begin{itemize}
        \item This is so the \textsc{Engineer} or \textsc{Technician} can trace the physical ``wiring'' (called traces) 
            that travel through the PCB to deliver power, ground or signals from one component to another on the PCB.
        \item The layout makes for good use of an \textit{ohmmeter}, to \textit{check continuity} from one place on 
            the PCB to another, ensuring signal integrity. Checking continuity means expecting and measuring a 0 ohm 
            \textit{short} reading on the ohmmeter from one side of a trace to the other.
        \item The layout diagrams describe where every component is and what it looks like in real life, layer by 
            layer if you held any component or subassembly in your hand and looked at it. In today's subscription-based 
            services, like Altium Designer, each of these layers can be inspected on the computer screen and highlighted 
            as needed for any circuit or trace on the PCB. These modern solutions to old paper manuals are discussed on 
            the next page.
    \end{itemize}
    \item Wiring and PCB Schematic diagrams.
    \begin{itemize}
        \item This is what the \textsc{Engineer} or \textsc{Technician} uses to identify electrical circuit 
            \\configurations for testing. When the circuit configuration is known, we can \\conduct tests and voltage 
            measurements with a ``good'' radio, then compare that result with a ``suspect'' or ``defective'' radio. 
            This is a very effective method of testing if there are more than one of the same assembly. This client was 
            given four HT1000s, so if one breaks, any radio repair facility can take one good unit and one broken unit, 
            and use the service manual to locate and fix the root cause of the failure.
    \end{itemize}
    \item Other useful tools in the service manual this author would use:
    \begin{itemize}
        \item How to test the complete assembly, what measurements to take and how to measure them.
        \item Pictures and part numbers of proprietary test equipment and proprietary tools for \\ purchase.
        \item Error Codes and how to retrieve, decode, interpret, and troubleshoot them.
        \item How to ``align'' the radio. This is a periodic maintenance item and done after some repairs.
            Radio alignment adjusts the radio so that it conforms to the FCC channel specifications for operation.
        \item How to disassemble and reassemble the radio and everything inside it.
        \item Maintenance procedures for the radio, just like a car.
        \item Exploded parts views and 3D renderings of physical parts, what they look like, and their part numbers.
    \end{itemize}
\end{itemize}

% === Page 8 ========================================================================
This old style service manual would typically be printed in a 3-ring binder and shipped to the Motorola radio dealers 
in addition to its use by the internal Motorola Engineering team. \\Nowadays, the manuals are electronic, and
the engineers who write the manual just send files to open in subscription-based computer programs or web tools 
to access schematics, layouts, and other information. In the next subsection there will be links to investigate 
some of these tools. Some are very, very expensive but usually have free trials for a month or so.

\begin{center}
    \subsection{Modern Service Manuals}
\end{center}

Some examples of modern engineering tools used to create the diagrams and layouts used for the content in
service manuals are now created with subscription based web tools or downloadable subscription based programs 
such as:

\medskip
\begin{itemize}
    \item \href{https://www.cadence.com/en_US/home/tools/pcb-design-and-analysis/orcad.html}{OrCad X}
    \begin{itemize}
        \item This software was originally used by the engineering firm this author worked at, but was 
            eventually abandoned for Altium Designer.
    \end{itemize}
    \item \href{https://www.altium.com/altium-designer}{Altium Designer}
    \begin{itemize}
        \item This is the software we used at the engineering firm in South Florida. This author used all 
            the files to test and repair the \textit{embedded computer systems} we manufactured at the 
            engineering firm.
    \end{itemize}
    \item \href{https://www.kicad.org/}{KiCad}
    \begin{itemize}
        \item This is 100\% Free and open-source software, and is very popular with students and hobbyists.
        \item This author has made projects with KiCad, and it is easy to use.
    \end{itemize}
\end{itemize}

These programs are all EDAs: Electronic Design Automation tools. They are far superior to drawing everything 
on paper and snail mailing it all like years ago.

\clearpage
% === Page 9 ========================================================================
\begin{center}
    \section{Channel Groups and Information}
\end{center}

What is considered a channel group? In this particular use case with the client's HT1000s, channels $1 - 5$ 
are considered a group by the author since they all have the same \textit{PL Tone}. Channel Group 1 functions 
with channel interoperability using a \textit{tone squelch} of 103.5 Hz when more than one radio 
is tuned to the same HT1000 channel. This feature blocks all other stations or interference from another station 
that is not using the same 103.5 Hz tone on their transmitter, since unmuting the receiving radio requires it. Also
note that when the radio receives traffic and unmutes, the red LED near the antenna will blink.
Later on, other commonalities of each group will be explained as they appear.

In the case of the wrong PL tone received by the radio but it hears an FM carrier signal on that channel, 
the red LED light near the antenna will still blink, indicating the presence of voice traffic, but the radio 
remains muted.

\begin{center}
    \item \subsection{Channel Group 1: Channels 1 - 5}
\end{center}

\medskip
\begin{enumerate}
    \item MURS 1
    \begin{itemize}
        \item Freq: 151.820 MHz
        \item Tone: 103.5 Hz
        \item Bandwidth: Narrow (11.25 kHz)
    \end{itemize}
    \item MURS 2
     \begin{itemize}
        \item Freq: 151.880 MHz
        \item Tone: 103.5 Hz
        \item Bandwidth: Narrow (11.25 kHz)
    \end{itemize}   
    \item MURS 3
    \begin{itemize}
        \item Freq: 151.940 MHz
        \item Tone: 103.5 Hz
        \item Bandwidth: Narrow (11.25 kHz)
    \end{itemize}
    \item MURS 4
    \begin{itemize}
        \item Freq: 154.570 MHz
        \item Tone: 103.5 Hz
        \item Bandwidth: Wide (20 kHz)
    \end{itemize}
    \item MURS 5
    \begin{itemize}
        \item Freq: 154.60 MHz
        \item Tone: 103.5 Hz
        \item Bandwidth: Wide (20 kHz)
    \end{itemize}
\end{enumerate}

\clearpage
% === Page 10 ========================================================================
\begin{center}
    \item \subsection{Channel Group 2: Channels 6 - 11}    
\end{center}

\medskip
\begin{enumerate}[resume]
    \item MURS 1
    \begin{itemize}
        \item Freq: 151.820 MHz
        \item Tone: 127.3 Hz
        \item Bandwidth: Narrow (11.25 kHz)
    \end{itemize}
    \item MURS 2
     \begin{itemize}
        \item Freq: 151.880 MHz
        \item Tone: 127.3 Hz
        \item Bandwidth: Narrow (11.25 kHz)
    \end{itemize}   
    \item MURS 3
    \begin{itemize}
        \item Freq: 151.940 MHz
        \item Tone: 127.3 Hz
        \item Bandwidth: Narrow (11.25 kHz)
    \end{itemize}
    \item MURS 4
    \begin{itemize}
        \item Freq: 154.570 MHz
        \item Tone: 127.3 Hz
        \item Bandwidth: Wide (20 kHz)
    \end{itemize}
    \item MURS 5
    \begin{itemize}
        \item Freq: 154.60 MHz
        \item Tone: 127.3 Hz
        \item Bandwidth: Wide (20 kHz)
    \end{itemize}
\end{enumerate}

\medskip
Note that all of these have the same tone of 127.3 Hz, so they are considered a channel group.
Channel 11 is slightly different than the rest of the channels in Group 2, although it should be 
interoperable with any of the HT1000s that are tuned to channel 10 or 11. This is discussed next.

\clearpage    
% === Page 11 ======================================================================== 
\medskip
The ``special'' channel 11 is exactly the same as channel 10, except at the end of every \\transmission on 
channel 11, there is a unique ``chirp'' noise that can be heard by the receiving radio. While listening, after 
the voice audio stops, the transmitting radio sends out a little data burst packet which is what many Police 
Departments used for unit ID. This data burst is decoded by the dispatch console and used typically by the 
dispatcher when one of the officers (or firefighters, EMS, etc.) calls out on the radio. The dispatcher knows 
immediately who is calling to keep transmissions shorter and help keep the channel clear of traffic. Each officer 
in a squad car will usually have two radios: one handheld, and one mobile unit mounted in the vehicle. Each radio 
can have its own unique ID, so the dispatcher knows if the officer is calling from the handheld or the squad car.

This chirp tone is named: \textit{MDC-1200}, and is set to ID on \textit{dekey}. For it to work properly,
the radio must be \textit{keyed up} for $\approx$ one second or longer.

\begin{enumerate}[resume]
    \item MURS 5 with MDC1200 ID on dekey.
    \begin{itemize}
        \item Freq: 154.60 MHz
        \item Tone: 127.3 Hz
        \item Bandwidth: Wide (20 kHz)
    \end{itemize}
\end{enumerate}

The purpose of having two separate channel groups with different tone squelch settings is for versatility. Since the
HT1000s have 16 available channels, rather than leave some of the channels unprogrammed, the author decided to create
two separate groups in case of interference. Channel group 1 will typically be sufficient for most operation, but
there is the possibility of being in a location where someone else is using the same tone frequency on their 
transmitter, since the MURS channels can be used by anyone in the general public. If in the rare instance MURS $1 - 5$
are busy with other traffic using the 103.5 Hz PL tone, channel group 2 can be used with a different PL tone of 
127.3 Hz to block out the unwanted traffic. This is not a perfect solution, though. If the other stations are 
operating close by, their FM carrier signals can still cause unwanted interference by blocking your wanted traffic
from being heard at all in some circumstances if their FM carrier signal is stronger. This can cause receiving 
operators to ``miss calls.''

\clearpage
% === Page 12 ========================================================================
\begin{center}
    \item \subsection{Channel Group 3: Channels 12 - 16}    
\end{center}

In Channel Group 3, all channels are set to receive (RX) only. Transmitting on these
channels is not allowed in this case due to the missing \textit{FCC Part 80} type certification on the HT1000 
radios, which is needed for transmitting on the Marine VHF band. Transmitting on the NOAA Weather Band is never 
allowed because the beacons are \textit{broadcast stations}.

\medskip
\begin{enumerate}[resume]
    \item Marine VHF 9
    \begin{itemize}
        \item Freq: 156.450 MHz (\textit{RX Only})
        \item Tone: \textit{Carrier Squelch}
        \item Bandwidth: Wide (25 kHz)
    \end{itemize}
    \item Marine VHF 16
     \begin{itemize}
        \item Freq: 156.80 MHz (RX Only)
        \item Tone: \textit{Carrier Squelch}
        \item Bandwidth: Wide (25 kHz)
    \end{itemize}   
    \item Marine VHF 22A
    \begin{itemize}
        \item Freq: 157.10 MHz (RX Only)
        \item Tone: \textit{Carrier Squelch}
        \item Bandwidth: Wide (25 kHz)
    \end{itemize}
    \item NOAA Weather Radio channel 3
    \begin{itemize}
        \item Freq: 162.475 MHz (RX Only)
        \item Tone: \textit{Carrier Squelch}
        \item Bandwidth: Wide (25 kHz)
    \end{itemize}
    \item NOAA Weather Radio channel 1
    \begin{itemize}
        \item Freq: 162.550 MHz
        \item Tone: \textit{Carrier Squelch}
        \item Bandwidth: Wide (25 kHz)
    \end{itemize}
\end{enumerate}

In the area of SE Massachusetts, the NOAA Ch 1 weather beacon is in the Boston area at the Blue Hills 
Observatory in Milton. The NOAA Ch 3 weather beacon is located in the Bourne/Hyannis area located at 
Camp Edwards. The stations self ID this information as well as its beacon \\transmitter callsign over the air
while listening. The National Weather Service shows more information about these beacons here:

\begin{itemize}
    \item \href{https://www.weather.gov/nwr/sites?site=KHB35}{NWS Ch 1: KHB35 Milton}
    \item \href{https://www.weather.gov/nwr/sites?site=KEC37}{NWS Ch 3: KEC37 Camp Edwards}
\end{itemize}

\clearpage
% === Page 13 ========================================================================
\begin{center}
    \item \subsection{Toggle Switch and Button Functions}   
\end{center}

\noindent Toggle Switch Definitions:

\medskip
\begin{itemize}
    \item Switch position A: Always High Power
    \item Switch Position B: Always Low Power
    \item Switch Position C: Channel Group Scan
    \begin{itemize}
        \item Group A: Scans all MURS channels by channel group of these HT1000s custom MURS \\
            channels when tuned to any of Channels 1 through 11
        \item Group B: Scans all 3 RX only Marine Channels on these programmed Radios.
                \begin{itemize}
                    \item Marine Channel 9
                    \begin{itemize}
                        \item Boater / General Call channel.
                    \end{itemize}
                \item Marine Channel 16
                    \begin{itemize}
                        \item USCG Call / S.O.S call channel.
                    \end{itemize}
                \item Marine Channel 22A
                    \begin{itemize}
                        \item USCG Announcements channel.
                    \end{itemize}
                        \begin{itemize}
                            \item USCG calls out on channel 16 for the announcement, then switches to channel 22A
                            to make the announcement.
                        \end{itemize}
                        \item When tuned to any Marine Channel on these Radios, the toggle switch can be switched to 
                            ``C'' and the radio will scan all 3 channels continuously until switched off of position 
                            ``C'', or if the channel is switched to another channel group.
                \end{itemize}
        \item Group C: Both local NOAA Weather Radio stations.
            \begin{itemize}
                \item Scan is deactivated on the 2 NOAA weather channels. This is because they are 
                    \textit{broadcast stations}, so they are continually transmitting and would not let the radio scan.
            \end{itemize}
    \end{itemize}
\end{itemize}

\medskip
Note that the Toggle Switch Configuration can be changed in the RSS (MS-DOS based). The only other useful
function we might care about in this application is changing the Toggle Switch Configuration to:
\begin{itemize}
    \item A: When the radio receives a signal, 2 conditions must be met to \textit{unsquelch} the radio.
    \begin{itemize}
        \item Sets the radio to sense an \textit{FM carrier} signal AND the PL Tone to unsquelch.
    \end{itemize}
    \item B: When the radio receives a signal, it only needs to hear the FM carrier to unsquelch.
    \item C: Channel scan function just as described earlier.
\end{itemize}

\medskip
All the other Toggle Switch functions in the RSS are mostly useless now since they are legacy configurations 
related to obsolete and outdated \textit{Trunking Protocols}. The other configurations can be viewed in the Radio
Service Software manual. The link for that manual is in Section 5 of this document. The side button configuration
menu can be viewed on page 88 of the RSS manual if interested.

\clearpage
% === Page 14 ========================================================================
\noindent Side and Top Button Definitions:
\begin{itemize}
    \item Side Button 1: Top round green button on the side of the radio.
    \begin{itemize}
        \item Function: Monitor
        \item When pressing this button momentarily, the radio will unsquelch and noise can heard on an unused channel.
            Holding the button for more than 5 seconds will result in the permanent unsquelching of the radio on that
            channel until the button is pressed again, or the radio is switched off and on again.
            \begin{itemize}
                \item This function is very handy for detecting weak stations that can be heard, but are too weak to
                    unsquelch the radio. Sometimes when just out of range of another station, the radio will not 
                    unsquelch, but pressing the monitor button may allow the operator to hear the weak station through
                    the noise.
                \item Note that when using the Monitor function, the tone squelch function is also \\overridden so any 
                    station, including environmental noise can be heard.
            \end{itemize}
        \item This function CAN be changed in RSS.
    \end{itemize}
    \item Side Button 2: Second from top button on the side of the radio.
    \begin{itemize}
        \item Function: Inoperative/Disabled.
        \item This function CAN be changed in RSS.
    \end{itemize}
    \item Side Button 3: Third from top button on the side of the radio.
    \begin{itemize}
        \item Function: Inoperative/Disabled.
        \item This function CAN be changed in RSS.
    \end{itemize}
        \item Side Button 4: Long bumpy oval button on the side of the radio.
        \begin{itemize}
            \item Push To Talk or ``PTT'' button.
            \item Function CANNOT be changed in RSS. This must always function as the PTT buttton.
        \end{itemize}
    \item Top Orange Button: Top of radio near antenna.
    \begin{itemize}
        \item Function: Inoperative/Disabled.
        \item This function CAN be changed in RSS.
        \item This function is typically used for an ``Emergency'' or ``Panic'' button on old trunking networks which 
            are all legacy and obsolete now, so it typically goes unused. In newer radios with a display, the orange
            button can typically be configured to light the display or toggle the high and low power setting in the 
            RSS. The ``Emergency'' function is only usable on trunking networks, otherwise it just creates 
            interference.
    \end{itemize}
\end{itemize}

\clearpage
% === Page 15 =========================================================================
\begin{center}
    \item \section{Other Thoughts}
\end{center}

In this guide, some different services in the VHF-Hi band for the range of 136 MHz $-$ 174 MHz were discussed,
since that is the frequency range of operation for the HT1000s.
In that band, there are many different services, some which were not discussed:

\medskip
\begin{itemize}
    \item Amateur Radio: 144 $-$ 148 MHz (FCC Part 97.)
    \item MURS band: 151 MHz and 154 MHz (FCC Part 95.)
    \item Marine VHF band: 156 MHz $-$ 157 MHz and 160 $-$ 162 MHz (FCC Part 80.)
    \item Railroad VHF: 159 MHz $-$ 161 MHz (Also FCC Part 90.)
    \item NOAA Weather Radio: 162 MHz (FCC Part 11 for Emergency Alert Systems.)
\end{itemize}

\medskip
Railroad VHF used to be conventional analog operation, but has mostly moved to digital \\operation, so the 
HT1000 would only hear the digital tones and not decode it properly for listening to voice traffic in this case. 
To decode digital voice traffic, a newer digital radio with a digital \textit{vocoder} is required.

\clearpage
% === Page 16 ========================================================================
\begin{center}
    \item \section{Links}
\end{center}

\medskip
Here is an explanation of the old television channel frequency allocations in VHF-Lo, 
VHF-Hi and UHF:
\medskip
\begin{itemize}
    \item \url{https://northpine.com/2023/03/15/explainer-vhf-low-vhf-high-and-uhf/}.
\end{itemize}

\medskip
Here is an online calculator that converts Frequency to Wavelength, along with some easy 
explanation: 
\begin{itemize}
    \item \url{https://www.everythingrf.com/rf-calculators/frequency-to-wavelength}.
\end{itemize}

\medskip
RadioReference is an excellent mostly free resource where most public service and Amateur Radio
frequencies, callsigns, and descriptions can be referenced. Here is a frequency allocation chart with 
public information on the MURS band:
\begin{itemize}
    \item \url{https://www.radioreference.com/db/aid/7733}.
\end{itemize}

\medskip
Here are the NOAA Weather Radio frequency allocations for the United States:
\begin{itemize}
    \item \url{https://wiki.radioreference.com/index.php/Weather_Radio}.
\end{itemize}

\medskip
Here are the Marine VHF boater and ship channel names and frequency allocations for the United States:
\begin{itemize}
    \item \url{https://www.navcen.uscg.gov/maritime-vhf-narrowband-channels}.
\end{itemize}

\medskip
Here is a chart of the complete FCC Part 97 band plan for US Amateur Radio:
\begin{itemize}
    \item \url{https://www.arrl.org/files/file/Regulatory/Band\%20Chart/Hambands4_Color_11x8_5.pdf}.
\end{itemize}

\medskip
Here is the complete FCC Online Table of Frequency Allocations. This covers all frequencies from 0 Hz up to 
3000 GHz and is several hundred pages long. This lists all services authorized to operate under FCC regulations,
where they are allowed to operate, and what type of RF emissions are approved for operation there.
\begin{itemize}
    \item \url{https://www.fcc.gov/sites/default/files/fcctable.pdf}
\end{itemize}

\medskip
HT1000 Service Manual for Jedi Series Radios:
\begin{itemize}
    \item \url{https://archive.org/details/manualsbase-id-372344}.
\end{itemize}

\medskip
HT1000 RSS User's Guide: This is also a free download from Repeater Builder, and it describes how to
use the Radio Service Software to configure and program the HT1000 radios. This provides a glimpse of different
options and configurations for the HT1000.
\begin{itemize}
    \item \href{https://manuals.repeater-builder.com/mo-files2/-HT1000/HT1000\%20RSS\%20Manual\%206881073C55-F.pdf}
    {HT1000 RSS User's Guide (URL too long.)}
\end{itemize}

\clearpage
% === Page 17 ========================================================================
\begin{center}
    \section{Vocabulary and Definitions}
\end{center}

\noindent For ease of reference, all references are in the page order they appear in this text.
\begin{center}
    \item \subsection{Page 3 Definitions}
\end{center}

\begin{enumerate}
    \item \textit{band plan}: A radio band plan (also called a frequency allocation plan or spectrum allocation 
        plan) is a structured framework that divides portions of the radio frequency spectrum into specific frequency 
        bands and assigns each band to particular services, users, or types of radio communications. It ensures orderly, 
        efficient, and interference-free use of the limited radio spectrum resource. In the United States, the 
        governing body which regulates this is the Federal Communications Commission (FCC). Some other examples 
        of separate band plans are as follows:
    \begin{itemize}
        \item US AM Radio Broadcast Band Plan: 540 $-$ 1700 kHz.
        \item US FM Radio Broadcast Band Plan: 88 $-$ 108 MHz.      
        \item US 700 MHz Cellular Band Plan: 698 $-$ 806 MHz.
    \end{itemize}
    \item \textit{MURS}: Multi Use Radio Service. The FCC channel allocations for operating here can be 
        seen on page 3. Operation here is covered under FCC Part 95 rules.
    \item \textit{FCC}: The Federal Communications Commission. This is the US Executive Branch agency that
        was created in the 1930s which provides the regulatory framework for frequency/channel allocations in the
        in the USA for ALL services. TV, radio, satellite, aircraft, ships, boats, everything. There are all 
        different ``FCC Part'' numbers for each different service.
    \item \textit{FCC Part 95}: This is the frequency allocations for Personal Radio Services (PRS). Part 95 
        includes MURS (license free), FRS (Family Radio Service, also license free), GMRS (General Mobile Radio 
        Service, requires a license and overlaps with FRS), and CB (Citizens Band, also license free).
    \item \textit{The Internet Archive}: This is a public archive of various materials available to anyone on 
        the internet for free. This archive has free books, manuals, games, and all kinds of other software.
    \item \textit{VHF}: Very High Frequency band. The entire band spans from 30 MHz to 300 MHz in frequency or
        10 meters to 1 meter in wavelength.
    \item \textit{VHF-Hi}: This is the portion of the VHF band from 100 MHz to 300 MHz, or 3 meters to 1 meter in
        wavelength.
    \item \textit{VHF-Lo}: This is the portion of the VHF band from 30 MHz to 100 MHz, or 10 meters to 3 meters
        in wavelength.
    \item \textit{UHF}: Ultra High Frequency band. The ``next highest'' band in frequency above VHF. UHF spans
        from 300 MHz to 3 GHz in frequency, or 1 meter to 10 centimeters in wavelength.

% === Page 18 ========================================================================    
    \item \textit{The Electromagnetic Spectrum}: This is the entire spectrum of frequencies in the universe. It
        spans from Radiofrequency waves (RF) at the lower end (kHz to several MHz), then to Microwaves (several GHz).
        Skipping ahead to the center of the Electromagnetic Spectrum is Visible Light. Just below Visible Light is 
        called Infrared, which is a type of light humans can't see (but the camera on your cellphone can!!). 
        Many TV remotes use infrared LEDs to control the TV. Night vision cameras also use infrared to create 
        the bright black and white night vision picture, like on doorbell cameras. Just above visible light are 
        Ultraviolet rays or UV Light, which we also can't see. At the very high end of the Electromagnetic Spectrum 
        are: X-rays, then Gamma Rays, and finally Cosmic Rays are the highest known frequency that exists in the 
        Universe. All of these have something in common: they are all particles and/or waves that can travel through 
        free space (around the universe in space). Their only difference is frequency. Sound waves are NOT in the 
        Electromagnetic Spectrum because they do not travel through free space, and cannot take the form of a 
        particle. Sound is considered a ``mechanical wave'' created by the motion of a speaker, and is therefore 
        not part of the Electromagnetic Spectrum. Mechanical Waves are still waves, though, so they can still be 
        measured in Hertz or Kilohertz. Compared to the frequency of Radio Waves, sound waves are MUCH lower 
        in frequency.
    \item \textit{FCC Part 97}: This is the rules of operation and frequency allocations for the Amateur Radio 
        (ham) Band.
    \item \textit{legacy}: This word refers to any software, hardware, license or etc. that is outdated and no
        longer used. Something that is considered ``obsolete'' is also considered a ``legacy item'' just like old
        Nintendo NES games from the 1980s.
    \item \textit{FCC Part 73}: These are the FCC regulations and spectrum allocation for all broadcast services, 
        including AM/FM radio and over the air (OTA) broadcast television.

\begin{center}
    \item \subsection{Page 4 Definitions}
\end{center}

    \item \textit{FCC Part 90}: This is the FCC allocation for ``Private Land Mobile Radio Service'' aka PLMR and
        covers licensed operation for services like: Police, Fire, City, County, Private Contractors, Crane Companies,
        and etc. that want to use high power radios. This is also the type certification for the HT1000 radios. They
        are ``FCC Part 90 Type Accepted'' for this type of use.
    \item \textit{channel allocation}: a channel allocation or frequency allocation is a specification that the
        FCC provides to operate in a certain radio service. This allocation will include specs like: maximum power
        for that channel, the channel bandwidth, and the frequency of the channel.
    \item \textit{Bandwidth}: Bandwidth is how much spectrum the radio uses on a specific channel. The FCC
        defines maximum bandwidth on MURS as either 11.25 kHz for MURS 1, 2 and 3, or 20 kHz
        for MURS 4 and 5. Even though the radio is tuned to MURS 1, for example, at 151.82 MHz, the
        radio needs to use a little spectrum on either side of that ``center frequency'' to send the voice traffic.
        Think of the FCC allocated ``max bandwidth'' as the total width of a lane on the road. In order to drive
        down the center of that lane, the car must be less wide than the lane, or the vehicle would not fit the lane,
        and you couldn't drive it on the road without interfering with oncoming traffic. It is the same principle in a
        radio transmitter. We must set the radio to transmit within its own ``lane'' so that it does not interfere 
        with other radio traffic on adjacent channels. The bandwith setting in the radio firmware (modified 
        via \textit{RSS}) must be set properly. The RSS has predetermined settings for channel bandwidth.
% === Page 19 ========================================================================
    \item \textit{RSS}: The Motorola trademark for ``Radio Service Software,'' which is the program used to 
        modify programmed radio data such as: operating frequency, PL tone, toggle switch \\configurations, and etc.
    \item \textit{ERP}: ERP stands for Effective Radiated Power. This value referes to how many watts of RF 
        power the transmitter is actually radiating from the antenna, which includes the total output power of the
        radio, minus transmission line losses, plus antenna gain (or loss). This value, in the case of the HT1000
        would be at least 5 watts on high power, which is the transmitter output rating. The transmission line 
        losses are nearly negligible since the antenna is directly connected to the radio. So the only factors that
        affect ERP on the HT1000 is: the power switch (low vs high power) and the antenna gain, measured in 
        decibels (dB.)
    \item \textit{interference}: According to the FCC, interference is defined as: ``The effect of unwanted energy 
        due to one or a combination of emissions, radiations, or inductions upon reception in a radiocommunication 
        system, manifested by any performance degradation, misinterpretation, or loss of information which could be 
        extracted in the absence of such unwanted energy.'' What does all that mean? Interference is any unwanted
        RF energy that affects signal reception. Interference can be accidentally caused by LED or fluorescent 
        lights, some electronic devices, or can be deliberate (and illegal) from another radio operator trying to 
        ``jam'' your signal so the receiving radio gets no signal or a degraded signal.
    \item \textit{interoperability}: This term refers to different radios, usually carried by different
        departments, which allow them to communicate beyond just their own dispatch unit. As an example: police, 
        fire and EMS could be on separate frequencies with separate dispatchers. However, each police, fire, 
        and/or EMS unit radio will have ``interoperability channels'' so they can talk to each other for whatever 
        reason when they need to, like if they were attending the same call. This goes up level by level typically, 
        where those same local units can have interoperability channels with neighboring local agencies or state 
        agencies, and state agencies usually have interoperability channels with federal agencies in case more than 
        one agency is responding to a call.

\clearpage
% === Page 20 ========================================================================
\begin{center}
    \item \subsection{Page 5 Definitions}
\end{center}

    \item \textit{Jedi Series}: The Motorola Jedi Series is a generation of radios that were in use in the 1990s
        throughout the 2010s or so. They were preceded by the Motorola ``Genesis Series'' (models like P200s and 
        HT600s), and proceeded by the ``Astro'' series (Astro Spectras and XTS3000s), which were then replaced by 
        the ``Astro25'' series of radios (XTS and XTL 5000s).

\begin{center}
    \item \subsection{Page 6 Definitions}
\end{center}

    \item \textit{Ohmmeter}: An ohmmeter is a device that measures ``resistance'' in Ohms ($ \Omega $).
        Resistance is a value that states how difficult it is for electricity to flow. Low ohms is easier for 
        electricity to flow, higher resistance is more difficult for electricity to flow.
    \item \textit{continuity check}: Performing a ``continuity check'' uses an ohmmeter. To do this, the circuit
        must be powered off, and having good continuity means having the lowest ohm value possible, like if we
        were measuring the continuity of a wire from the front of a vehicle to the rear.
    \item \textit{short or short circuit}: a short, or short circuit refers to a 0 $ \Omega $ value. This can
        be good or bad. It is bad to short any power source from positive directly to negative, like a car battery.
        This is dangerous, and causes sparks and can cause a fire. In the case of a single wire, we would expect 
        a 0 $ \Omega $ short from one end of the wire to the other, since it needs to deliver the power to some
        load or consumer (like the transmitter or a light bulb) with the least resistance possible.

\begin{center}
    \item \subsection{Page 7 Definitions}
\end{center}
    
    \item \textit{embedded computer systems}: An ``embedded computer system'' or ``embedded system'' is a system 
        that is dedicated to perform certain functions and do not typically have normal Graphical User Interfaces like 
        a laptop or a desktop computer. Some common examples of embedded computer systems are: an ATM machine, a 
        microwave, the computer system that runs the internal combustion engine in a car, or the entertainment system 
        in the vehicle. All these systems have the same thing in common: they all have dedicated purposes and no access 
        to the internal file system like a ``general use'' computer (laptop, desktop, etc.) The HT1000s are also 
        considered an embedded system.

% === Page 21 ========================================================================
\begin{center}
    \item \subsection{Page 8 Definitions}
\end{center}

    \item \textit{PL Tone}: ``PL'' is the Motorola trademark that means ``Private Line.'' This is somewhat of
        a misnomer, since there is no privacy guaranteed on the public airwaves. The following terms are functionally
        equivalent: PL Tone, Tone, and PL. Use whichever interchangeably. it just means the radio is also 
        transmitting an analog tone in addition to the voice traffic.
    \item \textit{tone squelch}: ``tone squelch'' is a setting on the radio that only allows the radio to unmute
        or ``unsquelch'' when it receives: 1) A carrier signal on the correct frequency and 2) the correct PL Tone. 
        This is different than the ``carrier squelch'' setting, which only requires a carrier signal on the correct 
        frequency to unmute the radio. There is another setting called ``DCS'' or ``Digital Coded Squelch'' which 
        functions equivalently to the tone squelch function, except instead of analog PL tones, it uses digital codes.
        Motorola uses its own trademarked name for DCS called ``DPL'' which means ``Digital Private Line,'' another
        misnomer, but functions equivalently to DCS.

\begin{center}
    \item \subsection{Page 10 Definitions}
\end{center}

    \item \textit{MDC-1200}: Officially named ``Motorola Data Communications 1200 \textit{baud}.'' This is one of 
        many forms of Automatic Number ID (ANI) that is used for unit identification. This MDC-1200 data burst is 16 
        bits of information, contained in four \textit{hexadecimal} digits. The range of this ID number is 
        0000 - FFFF. To calculate the total number of unique IDs available, we use the following formula:
        $$ 2^{\text{ }\# \text{ of bits}} = 2^{16} = 65536 \text{ total IDs.} $$
    \item \textit{baud}: ``baud'', also called ``baud rate'' just refers to how many binary bits (1s and 0s)
        are sent every second. $$ 1200 \text{ baud} = 1200 \text{ bits per second. VERY slow by todays standards.} $$
    \item \textit{hexadecimal}: Also called ``hex'' for shorthand, this is a ``base 16'' numbering system using 16 
        digits 0 $-$ 9 and letters A $-$ F. This is different than the traditional ``base 10'' decimal system we use
        everyday. In order to count in hex, we go like this:
    \begin{itemize}
        \item 0, 1, 2, 3, 4, 5, 6, 7, 8, 9, A, B, C, D, E, F, 10.
        \item 11, 12, 13, 14, 15, 16, 17, 18, 19, 1A, 1B, 1C, 1D, 1E, 1F, 20, and so on.
        \item Hex is just a convenient way to convert binary (1s and 0s) to something more human readable
        using numbers and a few letters. Note that 1 hex digit contains 4 binary bits.
    \end{itemize}
    \item \textit{dekey}: ``dekey'' refers to when the radio operator unkeys the radio, which means the
        operator releases the Push To Talk (PTT) button. Releasing the PTT button means the radio stops transmitting
        and reverts to receive mode.
% === Page 22 ========================================================================    
    \item \textit{keyed up}: A radio is ``keyed up'' when the PTT button is pressed. This puts the radio in
        transmit mode so the radio operator can talk over the air. This can also be referred to as the ``keyup'' 
        state of the radio, which is terminology referred to in the service and RSS manuals.

\begin{center}
    \item \subsection{Page 11 Definitions}
\end{center}

\item \textit{FCC Part 80}: The FCC set of rules that regulate marine, ship, and boater \\communications. Some newer
    Part 90 certified radios also hold Part 80 certification, such as some newer Motorola APX series radios. This is
    convenient for interoperability between local/state law enforcement and the US Coast Guard, draw bridge operators,
    and etc.
\item \textit{broadcast station}: A station or beacon that is always transmitting, like the NOAA weather stations,
    over the air (OTA) television channels, or AM/FM radio stations.
\item \textit{RX Only}: A channel that is set to ``listen only.'' These channels on a radio only receive and 
    are blocked from transmitting. In the case of the HT1000s, pressing the PTT button on an RX only channel will 
    result in an error tone emitting from the speaker. This warns the operator that transmitting (TX) is not allowed.
\item \textit{carrier squelch}: When the radio is not receiving any traffic, it remains muted or ``squelched'' so 
    the operator does not hear the static white noise that would be present on an unused channel. A radio set to 
    receive in ``carrier squelch'' mode will unmute for any signal that increases above the squelch level, including 
    noise and interference as long as it exceeds the squelch level. Why would this be desired? In the case of ham 
    radio, most of the author's radios are set to receive in carrier squelch mode so that ANY station calling in with 
    any PL tone or DCS code can be heard. Unfortunately, the disadvantage is receiving noise once in awhile. In most 
    commercial or public safety situations, carrier squelch mode is undesired, since the operators only want to hear
    traffic from their own group of operators. This is so no outside stations can break in easily and cause unwanted
    interference.

\begin{center}
    \item \subsection{Page 12 Definitions}
\end{center}

\item \textit{unsquelch}: When a radio is turned on, tuned to any channel, but not receiving a signal, no noise
    can be heard from the radio. This is the ``squelched'' state of the radio, so background static noise is
    not heard on an unused channel. When the radio receives voice traffic, it ``unsquelches'' or unmutes so 
    the audio can be heard from the speaker. As soon as the signal disappears, the radio returns to the squelched
    state, which is muted audio.
\item \textit{FM carrier}: FM stands for ``Frequency Modulation,'' which is the mode of transmission the HT1000 radios 
    use to transmit information (voice traffic) over the air. The FM Carrier is the signal that the receiving radio
    can detect so it can then unmute and the operator can listen to the voice traffic coming over the air. There are 
    two components to the radio transmission: the FM Carrier, which does not change and is used by the receiving radio
    to detect a signal, and the ``baseband,'' which is the actual voice traffic being sent over the air using the FM
    carrier. There are other modes of transmission, one of which may be familiar is AM, or ``Amplitude 
    Modulation'' which is a slightly different mode than FM. Explaining these modes is beyond the scope of this 
    document, but can easily be researched on the internet.

% === Page 23 ========================================================================   
\begin{center}
    \item \subsection{Page 13 Definitions}
\end{center}

\item \textit{Trunking Protocols}: Trunking protocols are standardized rules and procedures that govern 
    how trunked radio systems (also known as trunked Land Mobile Radio (LMR) or Logic Trunked Radio (LTR) systems)
    automatically manage and allocate a shared pool of radio frequency channels among multiple users or groups in 
    two-way radio communications. This is a common practice for public safety systems like Police, Fire, and EMS. In 
    conventional analog radio systems, such as MURS operation, each group of channels is assigned 
    a fixed, dedicated channel, which can lead to inefficiencies if the channel is idle or very busy and
    overloaded. Trunking uses a computer-controlled central system to dynamically assign available channels 
    on demand. This allows a large number of users, often far more than the actual number of physical 
    channels, to share the same limited spectrum efficiently, based on the statistical principle that not 
    all users transmit simultaneously. A trunking system can help public safety because it can prioritize
    traffic, like if the emergency button is pressed, and can help guarantee that any operator who needs a
    clear channel has one when needed. Amateur Radio does not use trunking, since it requires expensive and 
    unneccessary extra hardware, which is not needed for normal use. Non-trunked systems are considered 
    ``conventional systems'', such as using the HT1000s in the MURS band.

\begin{center}
    \item \subsection{Page 15 Definitions}
\end{center}

\item \textit{vocoder}: Vocoder is short for ``voice coder/decoder'' and is required for using digital two-way radio
    operation. Newer radios in the US for public safety, commercial and amateur \\communications use digital modes such
    as P25 (APCO 25) and DMR (Digital Mobile Radio) versus the analog modes like AM and FM. The primary purpose of a 
    vocoder is to efficiently convert analog human speech (the baseband) into a compressed digital format for 
    transmission over bandwidth-constrained digital radio channels, while enabling intelligible speech reconstruction 
    at the receiving end. This allows digital radios to achieve better spectrum efficiency, incorporate error 
    correction for clearer audio in noisy environments, and support additional features like encryption or data 
    integration compared to analog systems. Without a vocoder, voice signals would require much higher bitrates, 
    making them impractical for modern narrowband channels with bandwidths from 6.25 kHz $-$ 12.5 kHz in many digital 
    standards. This modern solution allows the division of the same available spectrum into smaller ``splinter'' 
    channels, allowing more overall channels and more users in the same spectrum allocation.

\end{enumerate}
\end{document}