
\documentclass[conference]{IEEEtran}
\IEEEoverridecommandlockouts
% The preceding line is only needed to identify funding in the first footnote. If that is unneeded, please comment it out.
\usepackage{cite}
\usepackage{amsmath,amssymb,amsfonts}
\usepackage{algorithmic}
\usepackage{graphicx}
\usepackage{textcomp}
\usepackage{xcolor}
\def\BibTeX{{\rm B\kern-.05em{\sc i\kern-.025em b}\kern-.08em
    T\kern-.1667em\lower.7ex\hbox{E}\kern-.125emX}}

\begin{document}

\title{From Crisis to Control: Sensor Networks and Smart Systems Shaping Effective Disaster Management \\
    \thanks{Special thanks to our families for their support throughout our journey.}
}

\author{
	\IEEEauthorblockN{
		Jehad Ismail\IEEEauthorrefmark{1},
		Joel Brigida\IEEEauthorrefmark{2},
		Barthody Alexandre\IEEEauthorrefmark{3},
		Stephanie Val\IEEEauthorrefmark{4} and
		Isabela Costa\IEEEauthorrefmark{5}}
	\IEEEauthorblockA{
		\IEEEauthorrefmark{1}
			College of Engineering and Computer Science\\
			Florida Atlantic University, Boca Raton, FL USA 33431\\
			Email: aismail2021@fau.edu }
	\IEEEauthorblockA{
		\IEEEauthorrefmark{2}
			Design and Prototyping\\
			Dark Ridge LLC, West Palm Beach, FL 33415\\
			Email: joel@joelbrigida.com }
	\IEEEauthorblockA{
		\IEEEauthorrefmark{3}
			College of Engineering and Computer Science\\
			Florida Atlantic University, Boca Raton, FL USA 33431\\
			Email: balexandre2019@fau.edu }
	\IEEEauthorblockA{
		\IEEEauthorrefmark{4}
			College of Engineering and Computer Science\\
			Florida Atlantic University, Boca Raton, FL USA 33431\\
			Email: sval2021@fau.edu }
	\IEEEauthorblockA{
		\IEEEauthorrefmark{5}
			College of Engineering and Computer Science\\
			Florida Atlantic University, Boca Raton, FL USA 33431\\
			Email: icosta2021@fau.edu }
}

\maketitle

\begin{abstract}

This paper explores the role of sensor networks and smart systems in effective disaster management. 
It discusses their contributions to prevention, preparation, response, and recovery phases, highlighting 
their transformative impact. The implementation of wireless sensor networks for landslide monitoring is 
presented as a case study and also touches upon the use of sensor networks in various natural 
disasters. This papare emphasizes the importance of emergency communication, remote monitoring through drones 
and cameras, and resilient infrastructure. Overall, it showcases the vital role of sensor networks and 
smart systems in proactive disaster management.\par

\end{abstract}

\begin{IEEEkeywords}
    Smart Systems, Sensor Networks, IoT, Embedded Systems\par
\end{IEEEkeywords}

\section{Introduction} % Section I

Sensor networks and smart systems play a crucial role in our modern world, offering a multitude of benefits
to society, with a primary focus on enhancing safety. Particularly in the field of disaster management, these 
technologies have emerged as indispensable tools. The utilization of specialized sensors designed to detect 
and respond to natural disasters, coupled with the implementation of alert systems, decision-making support, 
remote monitoring capabilities, and resilient infrastructure, has revolutionized the landscape of effective 
disaster management. Sensor networks and smart systems have truly transformed the way we approach and mitigate 
the impact of catastrophes, ensuring a more proactive and resilient response.\par

\section{Sensor Networks} % Section II: Jehad

Sensor networks and smart systems in disaster management serve four primary objectives: prevention, preparation, 
response, and recovery. These interconnected technologies play a pivotal role in each phase of disaster management, 
contributing to a more comprehensive and effective approach.\par

In reference to prevention, sensor networks and smart systems work proactively to prevent disasters or reduce their 
impact. By constantly monitoring environmental conditions such as the temperature, humidity, air quality, or 
seismic activity, sensor networks can allow for early warning signs and trigger alerts. With regard to 
preparation, sensor networks and smart systems assist in the preparation of disaster management. They can 
deliver real-time data and analytics that allow authorities to measure risks, plan response approaches, and 
allocate resources efficiently. In terms of response efforts, sensor networks and smart systems play a crucial 
role in response operations. They provide real time data on environmental conditions, infrastructure integrity, 
and the condition of the affected populations. This information will help emergency responders to make informed 
decisions, prioritize actions, and allocate resources to areas where they are needed the most. As for recovery, 
sensor networks and smart systems continue to assist in recovery efforts. They provide crucial data for damage 
assessment, structural health monitoring, and environmental recovery. Smart systems facilitate data analytics 
and decision support tools that assist in resource allocation, rebuilding strategies, and long-term recovery 
plans.\par

The utilization of sensor networks in monitoring natural disasters has garnered significant attention from 
researchers and engineers due to its widespread applicability. An intriguing case in point is the implementation 
of wireless sensor networks specifically designed for landslide monitoring. Kotta, et al \cite{je1} offer a 
compelling solution in the form of a wireless sensor network system that relies on accelerometers to detect 
vibrations associated with landslides. Through their experiments, they observed that when the accelerometer's 
value surpassed 1 gravity, it served as a critical indicator of substantial mass sliding and hazardous 
conditions.\par

This innovative wireless sensor network system demonstrates how technology can effectively contribute to 
disaster management. By employing accelerometers as sensing devices, the system can detect even minute changes 
in vibration levels, allowing for the early identification of potential landslides. The obtained data provides 
valuable insights into the intensity of mass sliding, enabling authorities to assess the severity of the 
situation and take appropriate measures to safeguard lives and property.\par

The study conducted by Kotta, et al \cite{je1} showcases the immense potential of wireless sensor networks 
in mitigating the risks associated with landslides. The implementation of such advanced monitoring systems 
not only enhances the accuracy of landslide detection but also improves the overall response time and 
decision-making during critical situations. Consequently, these findings pave the way for the development 
of more robust and efficient sensor network solutions that aid in the proactive management of natural 
disasters.\par

There are several types of natural disasters and each disaster uses sensor networks in its own unique ways. 
The use of sensor networks for landslides is solely just one example. With earthquakes, sensor networks are 
deployed to earthquake-prone areas to detect seismic activity and monitor the ground motion. Seismometers, 
accelerometers, and geophones are some of the sensors found in these networks that measure the intensity, 
duration, and frequency of the ground shaking. For floods, it`s pretty simple: water level sensors are placed 
in rivers, streams, and flood-prone areas and they continuously measure the water levels, predict flood events 
and issue warnings in real-time. With wildfires, there are smoke detectors and infrared sensors within the 
sensor networks that detect smoke and abnormal rise in temperature which can trigger alerts. Other technologies 
used include thermal cameras and remote sensors to monitor the behavior of fire, heat signatures, and patterns 
of the spreading fire. Pertaining to hurricanes and storms, weather monitoring stations use anemometers, 
barometers, and rain gauges within the sensor networks. These are responsible for measuring wind speed, 
atmospheric pressure, and precipitation. As for volcanoes, sensor networks are deployed near active volcanoes 
to monitor their volcanic activity. The types of sensors used here are seismic sensors and gas detectors 
and they are used to track ground vibrations, gas emissions, and any changes in volcanic activity. 
Collecting all this data allows for clear insights into volcanic eruptions. Lastly, for tsunamis Buoy-based 
sensors are installed in coastal waters to detect changes in sea level and to transmit real-time data. The 
Buoy-based sensors are integrated with seismometers that can detect underwater earthquakes which are 
associated with tsunamis. With these sensor networks in place, early warning systems can initiate evacuation 
procedures.\par

In general, sensor networks are instrumental in effective disaster management. They enable prevention through 
early warning systems, support preparation efforts by providing real-time data for risk assessment, facilitate 
response operations by offering critical information to emergency responders, and aid in the recovery phase by 
assessing damage and monitoring environmental conditions. With their ability to continuously monitor and 
collect data, sensor networks empower authorities to make informed decisions and take proactive measures, 
ultimately saving lives and minimizing the impact of disasters. As technology advances, sensor networks will 
continue to evolve, enhancing their role in disaster management and contributing to safer and more resilient 
communities.\par


\section{Emergency Communication And Alert Systems} % Section III: Isabela
There is, inherently, nothing more important for a sensor network to do when it has sufficient data than 
to be able to notify needed people about a—rather grave—situation. Take, for example, an alert system 
about mudslides, or tsunami's: those alerting systems are critical, of the utmost importance to save even 
one individual's life. The difference between a catastrophic event and one being minimized, could be done 
by a simple alarm, alerting those necessary about abnormal activity.\par

As discussed in Banjanovic-Mehmedovic, et al \cite{iz1}, an abnormal situation or anomaly is defined 
as a disturbance or series of disturbances in a process. For our examples, we'll be talking 
about how natural disasters or events of war can trigger these alarms. \par

For example, there are many mudslides in Brazil, and most of them are unfortunately deadly, even with 
alert systems in place. Brazil is a combination of poor city-planning on the loose 
terrain of a mountain, which creates grave results. When too much rain falls, the loose soil 
can't hold onto the weight of many houses that weren't built correctly on the mountain, 
and suddenly parts of the hillside wash away down the mountain. The most recent event was deadly and killed 
more than forty people located just outside Rio de Janeiro. This was an improvement before the Alert System
was installed because sirens blared, giving people a warning that there was potential for a mudslide, 
and urged them to evacuate.\par

How was this done? A combination of practicality and expertise in the area. The government does not have 
enough funding to create elaborate, precise mechanisms to warn their people, but they do have water 
bottles. Lots of them. And water bottles they used for their alarms. For the geologist that is well-known 
for creating alert systems within the country, using a water bottle to ``collect rainfall'', and ``when [the 
rain] reached a certain level, sirens blasted'' \cite{iz2}, is sufficient enough to create a system where people 
are able to escape their dangerous positions in time is more than useful. Even something as simple as a 
water bottle filling up to a certain level can be the difference between life and death— to go as far as 
``eighteen of Petropolis' 20 risk alert sirens sounded before Tuesday's fatal landslides'', \cite{iz2} which many 
civilians were aware of, warning them that there was going to be a disaster ahead. \par

Not only did they use sirens, but they also texted the people who lived in the area, ``[receiving] text 
messages from authorities, warning them about the coming storm'' \cite{iz2} that could impact their community. An 
alert system doesn't have to be just an alarm on a pole blaring noise, but it can also be texted or called 
to someone's phone, which can be more useful to those who have a disability—perhaps they can't hear—and 
having both options is more inclusive. Detection is important, but what good is detection if the people 
who can be affected don't know in time? These types of alarms are best known as conditional, as it 
would be wise if only certain people would get these notifications in order for the alarms to work 
more efficiently. If your house isn't built on the hillside in this city; you're away from any 
danger of having your house be swept away by the rain and floodwater, there is no sense in 
alerting you of imminent danger, as that would cause confusion. The alert system should be 
nuanced and take into consideration ``various points'' \cite{iz3}: GPS, weather advisory, 
population density, and, in our case, any filled water bottles that pass the reasonable threshold. \par

Our next example is the flooding that happened in Japan in 2011 that was caused by a tsunami. Tsunamis are 
created by earthquakes in the ocean that push large bodies of water onto land and can cause catastrophic 
damage to areas— as explored in the horrible tsunami in 2011, north-western Japan was ravaged by such 
weather. Luckily, there was hope for the residents, as the Japan Meteorological Agency issued the warnings 
and alerts to television and radio channels, the internet, and mobile phone networks \cite{iz3}, which is much 
more elaborate than the alert systems in Brazil. In as little as three seconds from the earthquake, 
warnings and alerts were issued to those who monitor such activity within the ocean floor to detect if 
there was a possibility of a natural disaster. A tsunami warning actually came out in twenty minutes after 
the actual earthquake. These timely, efficient ways to notify civilians helped safe many lives throughout 
this horrific event that could've been one of the worst in history, and they're not alone.
NOAA, a U.S. weather agency actually detected the earthquake as well and was able to warn the residents of 
Hawaii with similar alert systems. Buoys with long sensors, placed on the ocean floor, picked up 
disturbances that resembled earthquake patterns and were able to make their own calls on the subject \cite{iz4}. 
That wireless transmission to scientists in Hawaii, who later transferred it to local emergency managers 
who were able to use all of this information to deduce their next steps, was vital to keeping everyone 
safe, both on Japanese and American soil. Wireless transmissions need to be fast, speedy, with as much 
data as possible while being cost-effective. As we've previously discussed, it is not every country that 
can afford costly measures in keeping their citizens safe, and there needs to be a way to have good alarms 
yet being affordable. \par

There have been a few solutions, ranging from having a system of alarms that can alert a variety of things 
in one package  \cite{iz3}— perhaps an audible alarm system can have different noises depending on what they're 
alerting for, so a tornado alarm could be a different sound than an earthquake, though that would require 
the average human to recognize the different sounds. Nevertheless, the option shouldn't be tossed away so 
quickly; a hierarchy of alerts, with the architecture shaped in a way where the emergency alerts are given 
at a local scale, and not the entire city, is one to look into. As we've discussed, there's no need to 
alert the entire city, one that spans multiple kilometres across, about an imminent flood if sensors are 
only alerting to a specific hill or mountainside. It would be then useful for only a subset of the 
entirety of the alarm system to go off and alert citizens; likewise, it is more useful to alert the 
citizens that live in the tsunami-affected area of the dangers, instead of alerting the entirety of 
Japan.\par

\section{Decision Making Support} % Section IV: Alex

Technology improvement throughout the years does not need to be explained any longer because of how 
far we as a society have come. I.e., technologies such as Sensor networks and smart systems have played 
a crucial role in natural disaster management over the past few years and are continuing to do so 
by providing valuable data and decision-making support to minimize damages. Hence, this portion of 
the paper will discuss and explore some of the key points when it comes to sensor networks and smart 
systems decision-making support in natural disaster management and those key points utilization.\par

One of the main key outcomes of sensor networks and smart systems in relation to natural disaster 
management is providing life prevention warnings or to be more specific, “Early Warning Systems”. 
Sensor networks can be deployed in disaster-prone areas to detect early signs of potential natural 
disasters such as earthquakes, tsunamis, floods, and hurricanes. These sensors can monitor environmental 
parameters like seismic activity, water levels, wind speed, and rainfall, providing real-time data for 
early warning systems. Furthermore, Early Warning Systems are critical components of disaster management, 
they are designed to detect and alert communities or to be more specific, public disaster management 
workers about potential natural disasters before they occur or as they are happening. These systems help 
reduce the loss of life and property damage by providing timely and accurate information to authorities 
and the public.\par

Early Warning Systems can be categorized into multiple types such as Seismic and Tsunami Early Warning 
Systems which are systems that focus on earthquake-prone areas and coastal regions susceptible to tsunamis. 
Seismic sensors detect seismic activity, while buoys and tide gauges monitor sea levels for potential 
tsunamis; Wildfire Early Warning Systems which are deployed in areas prone to wildfires, these systems 
use remote sensing techniques, weather data, and fire monitoring tools to detect and predict the spread 
of wildfires; Meteorological Early Warning Systems which are systems that focus on weather-related 
disasters such as hurricanes, typhoons, storms, tornadoes, and heavy rainfall leading to floods. 
They use weather monitoring instruments like weather radars, satellites, and weather stations to 
predict and track meteorological events, etc.\par

As a result, to achieve these expectations, Early Warning Systems use Various types of sensors, 
collect large amounts of data, and analyze these data to make reliable decisions. \par

A second main key outcome of sensor networks and smart systems in relation to natural disaster management 
is data Collection and Monitoring. Smart sensor networks can collect vast amounts of data from 
disaster-affected areas. This data includes information on the intensity and spread of the disaster, 
infrastructure damage, and the affected population. Monitoring this data helps in assessing the 
situation accurately and allows for a swift response. Additionally, data collection and monitoring 
are fundamental aspects of natural disaster management. They involve the systematic gathering, 
processing, and analysis of various types of data to understand the ongoing situation, assess the 
impact of the disaster, and make informed decisions. Furthermore, the following paragraphs will 
dive more into the context of Data Collection and Monitoring in relation to disaster management.\par

Sensor networks and smart systems collect various types of data collection depending on the specific 
end goal that needs to be achieved. Examples of data collected are environmental Data which includes 
data related to meteorological conditions such as temperature, humidity, wind speed, and precipitation. 
It also encompasses hydrological data, such as river flow, water levels, and groundwater measurements. 
Next is seismic data that are collected by seismometers, which is crucial for monitoring earthquake 
activity; Geospatial Data which involves information related to the physical location and attributes 
of features on the Earth's surface. It includes topographic maps, satellite imagery, aerial photographs, 
and GIS (Geographic Information System) data; last but not least, Remote Sensing Data which are data such 
as satellite imagery and aerial surveys, provide valuable data for monitoring the extent of damage 
caused by natural disasters. They offer a comprehensive view of the affected regions, enabling 
decision-makers to prioritize response efforts effectively.\par

Furthermore, these data are monitored by data monitoring systems such as sensor networks which consist of 
a network of distributed sensors that continuously monitor various environmental parameters. These sensors 
can be deployed in disaster-prone areas to collect real-time data, enabling early warning and quick 
response. Real-time monitoring is another method that is used as a data monitoring system that provides 
live updates of data, allowing decision-makers to track the disaster's progression and make informed 
decisions in real time.\par

In addition, data are then required to be processed and analyzed through different methods such as 
Integrating data from various sources which is essential for creating a comprehensive picture of 
the disaster situation. Next, data are validated to ensure the accuracy and reliability of collected 
data because it is very critical that the data that is being collected is correct. Data validation 
involves checking for errors, inconsistencies, and outliers to ensure that decision-making is based 
on trustworthy information. Furthermore, data analytics and modeling techniques are applied to process 
and interpret the collected data. Predictive models use historical data and current sensor inputs to 
forecast the disaster's trajectory. As a result of data collection and monitoring, we are able to make 
informed decision-making, make impact assessments, and learn from past events.\par

Last but not least, we have remote sensing and imaging which are also one of the main key outcomes of 
sensor networks and smart systems in relation to natural disaster management. Smart systems can utilize 
remote sensing technologies, such as satellite imagery and drones, to assess the extent of damage caused 
by the disaster. These images provide valuable insights into the affected regions, which aids in 
prioritizing rescue and relief efforts. Remote sensing and imaging are powerful technologies that play 
a significant role in disaster management and natural resource monitoring. They involve the use of 
various sensors and imaging devices to collect data from a distance, typically from aircraft or satellites, 
about the Earth's surface and its environment. Here is a more in-depth look at remote sensing and 
imaging. Furthermore, the following paragraph will dive more into the context of remote sensing and 
imaging in relation to disaster management.\par

First, there are passive and active remote sensing. Passive remote sensing involves the detection of 
natural energy such as sunlight that is reflected, emitted, or scattered by objects on the Earth's 
surface. Passive sensors, such as cameras and radiometers, capture the energy to create images and 
spectral data. In contrast, Active remote sensing uses its own source of energy such as radar or lidar 
to illuminate the target and measures the reflected or scattered energy. This method enables remote 
sensing in all weather conditions and at night.\par

Second, there are spatial and temporal resolutions. The spatial resolution of satellite sensors refers 
to the level of detail in the imagery they capture. High-resolution sensors offer fine details, while 
low-resolution sensors provide broader coverage but with lower detail. In contrast, the temporal resolution 
represents the frequency at which a satellite revisits a specific area. Some satellites have high temporal 
resolution, allowing frequent updates, while others have lower temporal resolution due to their orbit and 
revisiting times.\par

Lastly, there are earth observation satellites which are satellites that are equipped with various sensors 
that capture data in different spectral bands, ranging from visible light to microwave and thermal infrared. 
They provide frequent and widespread coverage of the Earth's surface.\par

In the case of disaster management, these applications conduct damage assessments after a natural disaster 
using satellite imagery and can quickly assess the extent of damage to infrastructure, buildings, and 
vegetation. This information is invaluable for prioritizing response efforts; Monitoring environmental 
changes by monitoring changes in land use, deforestation, coastal erosion, and glacier movements, 
providing valuable data for environmental management and conservation; and conducting flood Mapping 
by utilizing satellite radar and optical sensors which capture flood extent and changes over time, 
supporting flood monitoring and management.\par

Moreover, besides early warning systems, data Collection and Monitoring, and remote sensing and imaging, 
there are many additional other key points that have strong utilization impacts on disaster manangment. 
These key points are real-time communication, post-disaster evaluation, and integration of AI and machine 
learning. For example, real-time communication facilitates seamless communication among emergency response 
teams, government agencies, and affected communities. Post-disaster evaluation continues to be valuable 
for conducting damage assessments and impact evaluations after a disaster by analyzing post-disaster 
data, decision-makers can learn from the event and improve future disaster response strategies. Lastly, 
integrating AI and machine learning algorithms can be utilized to analyze complex data patterns and make 
more accurate predictions. This enables decision-makers to have a deeper understanding of the disaster's 
potential impacts and make data-driven decisions.\par

Nevertheless, just as nothing is achieved overnight, there are currently many challenges with decision-making 
support in relation to Sensor networks and smart systems' role in natural disaster management. Some of these 
challenges are the need for stronger infrastructure, better data qualities and availability, and ethical 
consideration because during decision-making, we must be aware of how we save lives and on what fairground 
we approach such decisions.\par

Overall, early warning systems, data Collection and Monitoring, and remote sensing and imaging are instrumental 
in mitigating the impact of natural disasters, providing decision-makers and communities with critical 
time to prepare and respond effectively. Continuous improvement, integration of advanced technologies, 
and effective communication are key to enhancing the efficiency and effectiveness of these 
life-saving systems.\par

Conclusion. In conclusion, sensor networks and smart systems are powerful tools in natural disaster management. 
By providing real-time data, predictive insights, and efficient communication, these technologies empower 
decision-makers to make informed decisions and take timely actions to minimize damage and save lives 
during natural disasters. Furthermore, key points discussed earlier such as data collection and monitoring 
and remote sensing and imaging are indispensable components of natural disaster management. They enable 
decision-makers to gain critical insights, respond effectively, and minimize the impact of disasters 
on communities and infrastructure. Continuous improvements in data collection technologies, 
integration, and analysis are vital for enhancing disaster preparedness and response efforts.\par


\section{Remote Monitoring} % Section V: Steph/Specks

Remote Monitoring in disaster management involves using equipment such as cameras, drones, and the like, 
to survey affected areas, sensing temperatures, relative humidity levels, leaks, ventilation, dew points, 
adverse weather developments, and so on; Equipment that also relays ``information regarding power outages 
and weather changes ... [and] [t]he size of the equipment and components vary by manufacturer and 
model.'' \cite{b4} This equipment of sensors plays a focal role in the monitoring and alerting of disaster 
management, contributing to a more comprehensive approach in response to these disasters. With remote 
monitoring systems, recovery specialists and first responders can set considerations to maintain ideal 
circumstances in recovery efforts contributing to disaster readiness.\par

One example showing this is the application of drones, where drones have become an effective part of harm 
reduction, seeing that their use ``has rapidly evolved over the past decade ... [in] a variety of 
fields... and [has] becom[e] increasingly used in disaster management or humanitarian aid.'' \cite{b5} 
More specifically, the application of drones in disaster control has expanded to search and rescues in 
that they can ``reduce the time required to locate victims and the time required for subsequent intervention 
by searching a large area in a short period of time, ... providing critical information to rescuers about 
the route that needs to be taken. Additionally, ... searching for alive victims buried beneath rubble 
using sensors such as noise sensing, binary sensing, vibration, and heat sensing.'' \cite{b5} Thus, 
when it comes to disaster management, drones have significant potential in helping in searching for 
lost and trapped civilians due to disasters such as cave-ins, floods, hurricanes, and the like. Remote 
monitoring, with drones, has the potential to save the lives of people in high-risk areas, so they
can feel safer and more at ease, giving them a much higher chance of survival during a local disaster.\par

Another example of the potential benefits of remote monitoring in disaster management is found in 
landslide monitoring. Like in the use of drones, the Internet of Things (IoT) ``plays a major role for the 
purpose of monitoring natural disasters'' \cite{b7}, specifically in things like landslides; Seeing that 
landslides ``cause more than \$100 million in direct damage and cause thousands of fatalities'' \cite{b7}
and so, in ``order to mitigate the landslide hazard, several landslide monitoring techniques have been 
developed over the last decades'' \cite{b7}.  Remote sensing, a commonly used technique, is one of these 
methods, which is ``mainly used in landslide detection, fast characterization, and mapping 
applications'' to ``gather information about the distribution and kinematics of surface displacements. Remote 
sensing makes use of aircraft, spacecraft, or terrestrial-based platforms.'' \cite{b7} Even though the 
manual live supervision is limited due to the revisit period of satellites, it is typically ``suitable for 
mapping vulnerable areas ... [and] monitoring displacements over a large area ... with 3-D capabilities''.
\cite{b7} To elaborate further, these remote techniques collect data about an area using satellites, 
airborne, or ground-based sensors. ``[The] most commonly used techniques are based on laser, radar, and 
infrared sensor[s].'' \cite{b7} For example, these remote sensors, specifically terrestrial laser scanning 
(TLS), have been used for active landslides in the French Alps. Using automated monitoring to track 
rockfalls and landslides, providing near real-time monitoring/change detection for data collection. This 
automated monitoring helps find and predict areas at high risk of being affected and buried, allowing at-
risk people to be alerted and relocated when needed or for businesses to take the proper precautions.\par

An additional example of the potential benefits of remote monitoring in disaster management is also seen 
in Water Level Monitoring/Flood monitoring. Where the Internet of Things (IoT) ``plays a major role in 
the purpose of monitoring natural disasters;'' \cite{b7} like river behavior, and how it may ``help 
mitigate or prevent future disasters.'' \cite{b6} With that said, seeing that floods are amongst the 
``most common and devastating of all natural hazards [accounting for 41\% of all natural perils that occurred 
globally in the last decade]''. \cite{b3} Remote monitoring, correspondingly, has the potential to curb 
flood-related deaths and the cost of damages. And so, to mitigate the hazards of floods caused by things 
like water levels and storms, ``activities exploring how camera images and wireless sensor data...[that] 
can improve flood management''\cite{b3} have been developed and utilized. One method being computer-
vision, a commonly used technique based on cameras and ``relevant images from existing urban surveillance cameras 
[that] are captured and processed to improve decision-making.'' \cite{b3} These remote camera-based 
systems are more capable and commercial in that they ``involve low equipment cost and wide aerial 
coverage..., enabling the detection of flood levels at multiple points.'' \cite{b3} In other words, they 
have the advantage over traditional fixed-point methods of sensing in that they have an extensive 
reportage built on ``image processing techniques that have been widely applied in many fields, including aerospace, 
medicine, traffic monitoring, and environmental object analysis.'' \cite{b3} An aspect of how computer 
vision helps with disaster management of floods is how it aids with monitoring water levels in places such 
as lakes, rivers, and other potentially disastrous water sources since they are of extreme importance when 
it comes to early warning signs of a flood. Computer vision is useful in monitoring water levels with 
things like ``Image filtration...[which] plays a vital role in estimating water levels.'' \cite{b3} To 
elaborate further, a difference (image) method is utilized to analyze images of ``the region of interest 
(ROI) between the previous and current frame and then outputting a level of water...the water level is 
then estimated from the y-axis of the edged image.'' \cite{b3} This remote solution of difference has been 
utilized a few times and has shown to be dependable with adequate accuracy. And so, computer vision has 
been of great help and importance when it comes to disaster management and monitoring disasters caused by 
flooding with computers and cameras and has the budding to advance ``flood inundation mapping, debris flow 
estimation, and post-flood damage estimation[s].'' \cite{b3} \par

Overall, those are just some ways remote monitoring helps in disaster management of things like search 
and rescue, floods, and landslides. With the use of things like; drones, remote sensing, and computer 
vision, the equipment makes preventing the horrid outcomes of such disasters much easier.\par

\section{Resilient Infrastructure} % Section VI

% Smart Systems, especially for disaster recovery, require Resilient Infrastructure that incorporates 
% reliability, security (...) than just device security. Its very existence should not pose a threat to your  
% safety. We buy and drive cars and trust that at 70 MPH on the highway, the wheels won't fall off, or the
% steering wheel suddenly doesn't work. The vast new world of IoT devices must have ``SOMETHING'' so that
% they are trusted by the public. This can include everything down to where the minerals to make the
% batteries supplied with the device are mined and processed.\par

How is Resilient Infrastructure for Smart Systems in Disaster Management defined? Well, it may be
easier to define what it is not, since ``[t]he importance of resilience is continually increasing, as 
communication networks are becoming a fundamental component in the operation of critical 
infrastructures.'' \cite{jm1} Smart system infrastructure is not resilient if it is not able 
to continually provide needed data to its user base or response team for the prescribed period. 
This shows that battery or power source usage, as well as mode of communication and 
medium of data exchange need to be taken into account. \par

Building and deploying Resilient Infrastructure for smart systems, especially those aimed at Disaster
Management, can vary slightly depending on the type of natural disaster, for example, floods and 
hurricanes are handled differently than earthquakes, and the goal of the disaster management
task at hand: prevention, preparation, response, or recovery. There are other factors like 
risk analysis and risk management, discussed in Chang et al \cite{jm6}, which shows that proper 
Disaster Management is not ad hoc, it must be properly trained and budgeted for or the system 
will not be resilient. In general, though, there are many similarities. All resilient infrastructure 
are generally built with some redundancy at all levels, from redundant data storage to redundant
power supplies and even alternative power sources, like wind or solar it it is the only power
source available at any disaster location. Using redundancy in mission critical systems 
helps keep systems functional as a whole in case of one or more failures in the data chain from 
user to network.\par

In the landslide example discussed previously, the sensor network and its base station would likely 
need to be remote from any other base station that collects the data to be resilient. 
For infrastructure to be resilient in this case, since the operator or disaster response team may need 
to rely on streaming data from those sensors after the landslide
for days during disaster cleanup, the sensor network needs to have a sustainable power source that can last
through the prescribed amount of time. Depending on how far away 
each base station is from each other, the protocol for communication needs to be adjusted, along with the
required amount of power and frequency for data transmissions from the sensor network. 
In this case, since the sensor network is remote, if the data only needs to be transmitted 
from the sensor network site to a nearby location, less than one mile, then RF transceivers 
may work well. For those base stations to be connected to the internet poses a problem if the disaster 
zone is remote or lacking proper infrastructure, like in power outage areas. In all cases, Data Resilience
is important, as discussed in Vitaly, et al. \cite{jm1} Any time  \par

In other situations, for example, very isolated locations with no existing 
infrastructure, the best or only way for long range transmission may be communication via satellite. 
A traditional RF radio may not be desirable or usable if the distance between the system users or 
response team and the monitored disaster containing the sensor network are separated by more 
than a few miles, which is the approximate distance a high power transciever 
can transmit reliably without any radio towers in the viscinity for repeating or signal hopping so the 
data received is not corrupted. In general, higher frequencies like UHF (300 MHz to 3 GHz) and 
SHF (3 GHz to 30 GHz), which are where WiFi frequencies are located, offer an order of magnitude 
more spectrum, allowing extra channelized bandwitdth in a smaller space thanks to the shorter 
wavelength, however, the distance those higher frequencies travel without a signal repeater 
diminishes, but signals can be concentrated with dish antennas, as one example, to concentrate 
the RF into a directional beam much like a flashlight concentrates light in one direction. 
As frequency is lowered, the waves generally travel farther, but lower frequency bands 
like HF (3 MHz to 30 MHz) and VHF (30 MHz to 300 MHz) have much less room in the band to 
transfer wide bandwitdths of data. In addition to this, as discussed in Wireless Review, \cite{jm6}
the local terrain and type of foliage can have an effect on wireless transmissions at these
frequencies. \par

Resilient Infrastructure - How IoT can contribute to building resilient infrastructures that can withstand 
and recover from disasters. Infrastructures equipped with IoT sensors to trigger early warning systems. \par

Resilient Infrastructure also requires data as current as possible for disaster management smart systems, 
also.  or if you need a team that also requires data 
service to their laptops or gear and likely share network access to the devices being deployed \par 

One example of disaster recovery Resilient Infrastructure for disaster management uses 
``A wireless multihop infrastructure (WMI) [which] employs homogeneous and easily deployable
wireless routers to facilitate disaster-resilient Internet access in smart cities, emergency situations, 
and the Internet of Things. On the other hand, however, a WMI is subject to energy constraints, carries
unbalanced traffic, and has weaknesses in terms of maintenance, such as the lack of 
human management.'' \cite{jm2} WMI's use wireless mesh networks as base stations. Although not 
ideal for certain circumstances, the appeal of WMI type systems for disaster recovery is 
attractive since it is quickly deployable in mobile vehicles using small generators, and 
easily scalable to most natural disaster types. For devices deployed out in a remote location, 
they will need to be programmed with power saving and sleep modes so as not to drain the battery 
life before the prescribed period is expired. This aids in response time, and \par

Data Resilience \cite{jm1} needs to be discussed. Lose or corrupt the data that is transferred across the 
network, your IoT network is basically junk.\par

No matter your situation, though, power demands, data resiliency and data transmission to the end user 
or response team can be major hurdles. If the data being collected from the smart system only needs 
to get to a local user who is present can typically use an ad hoc network. If the system is remote 
from the users monitoring the system, then data transmission becomes the first issue. If the network 
is deployed in a city setting, then there is a likely 4G LTE or 5G Connection available to the 
network and the user, so data can be rapidly transferred. But if the disaster is in a jungle 
or a desert or somewhere remote, then options are limited. A system can be deployed into a remote 
location, and record data for a certain period of time and then the user returns to gather the data 
to analyze at a future time or date.\par

The environmental conditions for deployed disaster relief smart systems are an important factor for 
resilient infrastructure. Any sensors or smart systems subjected to outdoor weather need protection 
from extreme weather. For example: a sensor or smart system deployed in a hot, sunny climate needs 
to be tolerant of heat and sunlight. A smart system deployed to an arctic climate must be built 
to a different specification for cold tolerance and water resistance. Temperature also has an
effect on power sources, too. Extreme hot or cold conditions are generally not favorable for
systems that use batteries, so they may have a higher failure rate and need more frequent
battery replacement. \par

\section{Conclusion} % Section VII: Joel


The integration of sensor networks and smart systems with Disaster Management has allowed humanity
to monitor, prevent, prepare for, respond to, and recover from disaster much more efficiently, quickly, and 
safely, putting us in more control to mitigate the overall effects on our way of life. Disaster Management 
Smart Systems have proven themselves as valuable tools for natural disasters. Some key aspects of 
disaster management smart systems are implementation of the alert systems (example), decision-making 
support (example), remote monitoring capabilities (example), and resilient infrastructure (example).\par

When Sensor Networks for Disaster Management are deployed, they provide their users with data-rich 
environments that allow as close to real-time monitoring as possible to assist in disaster prevention
and/or mitigation, like the landslide example. Together with Emergency Communications and Alert
systems, proper action can be taken by authorities, including staging a response crew in preparation, 
dispatching a response team after a disaster, or providing an early warning to residents in the
disaster zone. Remote Monitoring enables real-time analysis and reaction from some point of 
safety local to the disaster, or halfway across the world if the internet or satellite uplink 
is available. Finally, resilient infrastructure was discussed that ensures the deployed system
is able to withstand and operate after the disaster it is monitoring, and provide redundancy 
and fault tolerance so that system operation can fully function in case of partial failure 
of the system.\par

Various disaster scenarios are studied and presented, and as the landscape for disaster recovery
smart systems becomes more available at a cheaper cost, the implementation of these systems 
becomes easier and helps build a safer and more aware, prepared community. \par


% Bibliography Entries are all at the end in IEEE Template...
\begin{thebibliography}{00}
    % Jehad: Sensor Networks Begin
    \bibitem{je1} Herry Z Kotta, Kalvein Rantelobo, Silvester Tena, and Gregorius Klau, ``Wireless sensor
        net-work for landslide monitoring in nusa tenggara timur,'' TELKOMNIKA, 
        (TelecommunicationComputing Electronics and Control), 9(1):9-18, 2011
    
    % Isabela: Alert Systems Begin
    \bibitem{iz1}L. Banjanovic-Mehmedovic, M. Zukic, and F. Mehmedovic, “Alarm detection and monitoring in 
    industrial environment using hybrid wireless sensor network,” SN Applied Sciences, vol. 1, no. 3, 
    Feb. 2019, doi: https://doi.org/10.1007/s42452-019-0269-y.
    
    \bibitem{iz2} ``Experts say Brazil's deadly mudslides reflect poor planning in the face of climate change,'' 
        PBS NewsHour, Feb. 18, 2022. https://www.pbs.org/newshour/world/experts-say-brazils-deadly-mudslides-
        reflect-poor-planning-in-the-face-of-climate-change
    
    \bibitem{iz3} D. G. Costa, F. Vasques, P. Portugal, and A. Aguiar, “A Distributed Multi-Tier Emergency 
        Alerting System Exploiting Sensors-Based Event Detection to Support Smart City Applications,” Sensors, vol. 20, no. 1, p.170, Dec. 2019, doi: https://doi.org/10.3390/s20010170.
    
    \bibitem{iz4} “Tsunami Strike Japan - Warning Systems,” oceantoday.noaa.gov.
        https://oceantoday.noaa.gov/fullmoon-tsunamistrikewarning/welcome.html

    % Need to organize these:    
    \bibitem{b2} D. Prasad, A. Hassan, D. K. Verma, P. Sarangi and S. Singh, ``Disaster Management System
        Using Wireless Sensor Network: A Review,'' 2021 International Conference on Computational 
        Intelligence and Computing Applications (ICCICA), Nagpur, India, 2021, pp. 1-6, 
        doi: https://doi.org/10.1109/ICCICA52458.2021.9697236.
    
    \bibitem{b3} Arshad, Bilal, et al. ``Computer Vision and IoT-Based Sensors in Flood Monitoring and
        Mapping: A Systematic Review.'' Sensors, vol. 19, no. 22, 16 Nov. 2019, p. 5012, 
        doi: https://doi.org/10.3390/s19225012. Accessed 30 Sept. 2020.

    \bibitem{b4} ``How Disaster Recovery Teams Use Remote Monitoring.'' Www.polygongroup.com, 
        www.polygongroup.com/en-US/blog/how-remote-monitoring-services-assist-disaster-recovery-teams/. 
        Accessed 30 June 2023.

    \bibitem{b5} Mohd Daud, Sharifah Mastura Syed, et al. ``Applications of Drone in Disaster Management: 
        A Scoping Review.'' Science and Justice, vol. 62, no. 1, 1 Jan. 2022, pp. 30-42, 
        www.sciencedirect.com/science/article/pii/S1355030621001477, 
        doi: https://doi.org/10.1016/j.scijus.2021.11.002.
    
    \bibitem{b6} Moreno, Carlos, et al. ``RiverCore: IoT Device for River Water Level Monitoring over
        Cellular Communications.'' Sensors, vol. 19, no. 1, 2 Jan. 2019, p. 127, 
        www.ncbi.nlm.nih.gov/pmc/articles/PMC6338933/, doi: https://doi.org/10.3390/s19010127.
    
    \bibitem{b7} Thirugnanam, Hemalatha, et al. ``Review of Landslide Monitoring Techniques with IoT 		
        Integration Opportunities.'' IEEE Journal of Selected Topics in Applied Earth Observations and Remote 
        Sensing, vol. 15, 2022, pp. 5317-5338, https://doi.org/10.1109/jstars.2022.3183684. Accessed 31 July 
        2022.
    
% Decision Making Support Begin (Alex)
    \bibitem{al1} Kuglitsch, M. (2022) Artificial Intelligence for Disaster Risk Reduction: Opportunities, 
        challenges, and prospects, World Meteorological Organization. Available at: 
        https://public.wmo.int/en/resources/bulletin/artificial-intelligence-disaster-risk-reduction-opportunities-
        challenges-and(Accessed: 25 July 2023). 

    \bibitem{al2} Andronie, Mihai. "Artificial Intelligence-Based Decision-Making Algorithms, Internet of 
        Things Sensing Networks, and Deep Learning-Assisted Smart Process Management in Cyber-Physical Production 
        Systems." Mdpi.Com, 21 Oct. 2021, www.mdpi.com/2079-9292/10/20/2497. Accessed 25 Jul. 2023.

    \bibitem{al3} Wu, Qi. "Applications and Theoretical Challenges in Environmental Emergency Issues Alerting 
        System on IoT Intelligence." Science Direct, 24 Apr. 2020, 
        www.sciencedirect.com/science/article/pii/S0140366420301717. Accessed 25 Jul. 2023.

    \bibitem{al4} Khalifeh, Ala’. "Chapter 16 - Smart Remote Sensing Network for Early Warning of Disaster 
        Risks." Science Direct, 17 Jun. 2022, www.sciencedirect.com/science/article/abs/pii/B9780323911665000124. 
        Accessed 25 Jul. 2023.

    \bibitem{al5} Konečný, Milan. "Early Warning and Disaster Management: The Importance of Geographic  
        Information (Part A)." Taylor and Francis Online, 20 Aug. 2010, 
        www.tandfonline.com/doi/full/10.1080/17538947.2010.508884. Accessed 25 Jul. 2023.

% Remote Monitoring Begin: Steph/Specks

% JB: Resilient Infrastructure Begin
    \bibitem{jm1} K., Vitaly. ``Data Resilience in Large-Scale IOT Deployments.'' Forbes, 21 Apr. 2022, 
        www.forbes.com/sites/forbestechcouncil/2021/11/19/data-resilience-in-large-scale-iot-deployments/?sh=1a1f7362c88c. 
	
    \bibitem{jm2} R. Mohanon, ``What Is Edge Computing? Components, Examples, and Best Practices,''
        Spice Works, Feb. 2022, Accessed: Jul. 20, 2023. [Online]. 
        Available: https://www.spiceworks.com/tech/edge-computing/articles/what-is-edge-computing/
    
    \bibitem{jm3} A. Sinha, P. Kumar, N. P. Rana, R. Islam, and Y. K. Dwivedi, 
        ``Impact of internet of things (IoT) in disaster management: a task-technology fit perspective,''
        Annals of Operations Research, Oct. 2017, doi: https://doi.org/10.1007/s10479-017-2658-1.

    \bibitem{jm4} R. Teng, H. -B. Li, B. Zhang and R. Miura, ``Differentiation Presentation for 
        Sustaining Internet Access in a Disaster-Resilient Homogeneous Wireless Infrastructure,'' in 
        IEEE Access, vol. 4, pp. 514-528, 2016, doi: 10.1109/ACCESS.2016.2519244.
    
    \bibitem{jm5} S. E. Chang, T. McDaniels, J. Fox, R. Dhariwal, and H. Longstaff, 
        ``Toward Disaster-Resilient Cities: Characterizing Resilience of Infrastructure Systems with 
        Expert Judgments,'' Risk Analysis, vol. 34, no. 3, pp. 416-434, Oct. 2013, 
        doi: https://doi.org/10.1111/risa.12133.

    \bibitem{jm6}
        T. Kridel, ``Less Than Stellar,'' Wireless Review, Jan. 1999, Accessed: Jul. 25, 2023. [Online]. 
        Available: https://link.gale.com/apps/doc/A53641879/AONE
\end{thebibliography}
    \vspace{12pt}
\end{document}