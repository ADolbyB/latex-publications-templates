% !TeX root = test.tex
\documentclass[letterpaper]{article} % Set Paper Size to 8.5 x 11 Letter Size
\usepackage[english]{babel}
\usepackage[margin=1in]{geometry} % Set Margins to 1 inch standard
\usepackage[nottoc]{tocbibind} % Includes "References" In the table of Contents
\usepackage{amsmath,amssymb} % Math Libraries

% Title, Date, and Author of the Document:
\title{Bibliography Management: BibTex}
\author{Joel M. Brigida}

%Beginning of the Document
\begin{document}

\maketitle

\pagebreak

\tableofcontents

\pagebreak

\section{First Section}
This document is an example of BibTex usage for Bibliography Management.
Three items are cited: \textit{The \LaTeX\ Companion} book \cite{latexcompanion},
the Einstein journal paper \cite{einstein}, and Donald Knuth's website \cite{knuthwebsite}.
The \LaTeX\ related items are \cite{latexcompanion,knuthwebsite}.
Another additional citation is \cite{dummy}, which is a book without a title.

\medskip

\section{Second Section}
How To Use LaTeX in VS Code
\begin{enumerate}
    \item No need to install Perl and latexmk.
    \item Install LaTeX Workshop VS Code Extension.
    \item Install MikTex on your system.
    \item We are good to go \dots
\end{enumerate}

\section{Third Section}

\noindent Standard \LaTeX{} practice is to write inline math by enclosing it between \verb|\(...\)|:

\begin{quote}
In physics, the mass-energy equivalence is stated 
by the equation \(E=mc^2\), discovered in 1905 by Albert Einstein.
\end{quote}

\noindent Instead of writing (enclosing) inline math between \verb|\(...\)| you can use \texttt{\$...\$} to achieve the same result:

\begin{quote}
In physics, the mass-energy equivalence is stated 
by the equation $E=mc^2$, discovered in 1905 by Albert Einstein.
\end{quote}

\noindent Or, you can use \verb|\begin{math}...\end{math}|:

\begin{quote}
In physics, the mass-energy equivalence is stated 
by the equation \begin{math}E=mc^2\end{math}, discovered in 1905 by Albert Einstein.
\end{quote}

\medskip

\pagebreak

\section{Fourth Section}
Here are examples of equations:

\noindent This requires \verb|\usepackage{amsmath,amssymb}|

\begin{equation}
    H_p(q) = -\frac{1}{N} \sum_{i=1}^{N} y_i \cdot \mathrm{log}_{10}(p(y_i)) + (1-y_i) \cdot \mathrm{log}_{10}(1-p(y_i))
\end{equation}

\begin{equation}
\chi_{\mathbb{Q}}(x)=
    \begin{cases}
        1 & \text{if } x \in \mathbb{Q}\\
        0 & \text{if } x \in \mathbb{R}\setminus\mathbb{Q}
    \end{cases}
\end{equation}

\begin{equation}
    \mathrm{Prediction:} \ \varphi(v) = \begin{cases} 
        1, & \mathrm{if} \ v\ge 0 \\
        0, & \mathrm{if} \ v < 0 
     \end{cases}
\end{equation}

\begin{equation}
    \mathrm{Forward \ Propagation:} \ \sum_{i = 0}^{m} w_i x_i = v = \vec{w} \cdot \vec{x}
\end{equation}

\begin{equation}
    w_1x_1 + w_2x_2 + w_0 = 0 \ \Longrightarrow \ x_2 = \frac{-(w_1x_1 + w_0)}{w_2}
\end{equation}

\medskip

\pagebreak

% Set the bibliography style to ieeetr and import
% the bibliography file "bibtexbib.bib",
\bibliographystyle{ieeetr}
\bibliography{bibtexbib}

\end{document}