\documentclass[11pt]{article}
\usepackage[margin=1.0in]{geometry} % Set Margins to 1 inch standard
\usepackage{hyperref} % use hyperlinks
\usepackage{indentfirst} % Set 1st paragraph indentation
\pagestyle{empty} % Turn Off Page Numbering

\renewcommand{\baselinestretch}{1.25} % set default paragraph line spacing

\title{\LaTeX\ Tutorial 5: Text and Document Formatting}
\author{Joel M. Brigida: ADolbyB}
\date{xXx Any Date xXx} % leave blank for no date or \today for current date.

\renewcommand{\contentsname}{Table of Contents:} % Custom name for Table of Contents

\begin{document}
\maketitle % Document Title
\tableofcontents
\thispagestyle{empty} % Remove Stubborn 1st Page Number

\pagebreak

\section{Text Formatting:}

This will produce \textit{italicized} text.

This will produce \textbf{Bold Face} text.

This will produce \textsc{Small Caps} text.

This will produce \texttt{typewriter monospace} text.

\hspace{\parindent}Example: Visit my website (not a link): \texttt{https://joelbrigida.com}

Make a custom hyperlink (requires hyperref pkg): \href{https://joelbrigida.com}{My Website}

To show just the web address as a hyperlink: \url{https://joelbrigida.com}

Change font of hyperlink (requires hyperref pkg): \texttt{\href{https://joelbrigida.com}{My Website}}


\section{Font Sizing:}

PEMDAS: Please Excuse My Dear Aunt Sally. (Normal Size)

PEMDAS: Please Excuse My \begin{large}Dear Aunt Sally\end{large}. (large)

PEMDAS: Please Excuse My \begin{Large}Dear Aunt Sally\end{Large}. (Large)

PEMDAS: Please Excuse My \begin{huge}Dear Aunt Sally\end{huge}. (huge)

PEMDAS: Please Excuse My \begin{Huge}Dear Aunt Sally\end{Huge}. (Huge)

PEMDAS: Please Excuse My \begin{normalsize}Dear Aunt Sally\end{normalsize}. (normalsize)

PEMDAS: Please Excuse My \begin{small}Dear Aunt Sally\end{small}. (small)

PEMDAS: Please Excuse My \begin{scriptsize}Dear Aunt Sally\end{scriptsize}. (scriptsize)

PEMDAS: Please Excuse My \begin{tiny}Dear Aunt Sally\end{tiny}. (tiny)


\section{Text Justification:}
\begin{center}
    This Line Is Centered (center).
\end{center}

\begin{flushleft}
    This Line is Left-Justified (flushleft).
\end{flushleft}

\begin{flushright}
    This Line Is Right-Justified (flushright).
\end{flushright}

\pagebreak

\begin{center}
    \section*{Example Sections and Subsections}
\end{center}

\section{Linear Functions}
    Linear Functions are generally defined in rectangular coordinates. These are functions which define straight
lines. The rectangular coordinate plane is also called the Cartesian coordinate plane after Rene Descartes. Below are
several exxamples of different linear equation types.
    \subsection{Slope-Intercept Form}
        \subsubsection{Example 1:}
            $y = mx + b$
        \subsubsection{Example 2:}
            $3 = 8x + 5$
    \subsection{Standard Form}
        $ Ax + By = C $
    \subsection{Point Slope Form}
        $ y - y_1 = m(x - x_1) $
\section{Quadratic Functions}
    Quadratic functions are generally defined in rectangular coordinates. These are functions which define a
parabolic curve, which is in the shape uf a parabola. A true parabola is only defined as opening up or opening down.
A parabola which opens to the right or left is acually made up of 2 functions: the positive and negative square
root function.
    \subsection{Vertex Form}
        $ (x - h^2) + k = 0 $
    \subsection{Standard Form}
        $ Ax^2 + Bx + C = 0 $
    \subsection{Factored Form}
        $ (x - p)(x - q) = 0 $

\end{document}