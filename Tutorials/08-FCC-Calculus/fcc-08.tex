\documentclass[11pt, letterpaper]{article}
\usepackage[margin=1in]{geometry}
\usepackage{amsfonts, amssymb, amsmath}

\renewcommand{\baselinestretch}{1.5} % set default paragraph line spacing

\title{\LaTeX\ Tutorial 8: Calculus Notation}
\author{Joel M. Brigida: ADolbyB}
\date{} % leave blank for no date or \today for current date.

\begin{document}
\maketitle
\thispagestyle{empty} % Remove Title Page Number

\pagebreak
\setcounter{page}{1} % Make Page #1 after the title page.
% Note that using the \begin{equation} method automatically puts the equation in 'display mode (larger).
% For inline display mode, wrap the equation in \diplaystyle{}

The function $f(x) = (x-3)^2 + \frac{1}{2}$ has domain $\mathrm{D}_f:(-\infty, \infty)$ and range $\mathrm{R}_f:\left[\frac{1}{2}, \infty\right)$
\begin{equation}
    \lim \limits_{x \to a^-} f(x)
\end{equation}

\begin{equation}
    \lim \limits_{x \to a^+} f(x)
\end{equation}

\begin{equation}
    \lim \limits_{x \to a} \frac{f(x) - f(a)}{x-a} = f'(a)
\end{equation}

\begin{equation}
    \int \sin(x) \, \mathrm{d}x = -\cos(x) + C
\end{equation}

\begin{equation}
    \int_{a}^{b} f(x) \, \mathrm{d}x % notice the limits are to the right of the integral sign
\end{equation}

\begin{equation}
    \int \limits_a^b f(x) \, \mathrm{d}x
\end{equation}

\begin{equation}
    \int \limits_{a}^{b} x^2 \, \mathrm{d}x = \left[\frac{x^3}{3}\right]_{a}^{b} = \frac{1}{3}b^3 - \frac{1}{3}a^3
\end{equation}

\begin{equation}
    \sum \limits_{n=1}^{\infty}ar^n = a + ar + ar^2 + \cdots + ar^n
\end{equation}

\begin{equation}
    \int \limits_a^b f(x) \, \mathrm{d}x = \lim \limits_{x \to \infty} \sum \limits_{k=1}^{n} f(x_k) \cdot \Delta x
\end{equation}

\begin{equation}
    \vec{v} = v_1 \vec{i} + v_2 \vec{j} = \langle v_1, \, v_2 \rangle
\end{equation}

\end{document}