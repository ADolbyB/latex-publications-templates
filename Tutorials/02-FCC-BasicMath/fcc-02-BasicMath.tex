\documentclass[11pt]{article} % Can Also be 'book' or 'report'
\usepackage[margin=1in]{geometry} % Set Margins to 1 inch standard
\usepackage{amsmath, amssymb, amsfonts}
\parindent 0px % Turn Off Paragraph Indentation
\pagestyle{empty} % Turn Off Page Numbering

\begin{document}
\title{\LaTeX\ Tutorial 2: Basic Math Notation}
\author{Joel M. Brigida: ADolbyB}
\maketitle % Document Title

\section{Superscripts:}
\thispagestyle{empty} % Remove Stubborn 1st Page Number

$$2x^3$$

More than 1 character in the exponent: $$2x^{34}$$

Functions in the exponent: $$2x^{3x+4}$$

Power to a Power exponents: $$2x^{3x^{4}+5}$$

\section{Subscripts:}

$$x_1 + x_{12}$$

Subscript in a subscript: $$x_{1_{12}} + y_{1_{2_{3}}}$$

Series: $$a_0 + a_1 + a_2 + \ldots + a_n$$

\section{Greek Letters:}

Some Popular Examples: $$\pi \ \Pi \ \alpha \ \varepsilon $$

Equations With Greek Letters: Area of a circle:$$A = \pi r^{2}$$

\section{Trigonometric Functions:}

Some Popular Examples: $$y = \sin(x) $$
$$y = \cos(\theta)$$
$$\theta = \tan^{-1} \left( \frac{y}{x} \right) $$
$$\theta = \arcsin \left( \frac{y}{r} \right) $$

\section{Log Functions:}

$$ \text{Common Log (Base 10): } y = \log (x) $$
$$ \text{Log Base 2 (Binary): } y = \log_{2}(x) $$
$$ \text{Log Base } e \, \text{(Natural Log): } y = \ln (x) $$

Another Way:

\begin{center}
    Common Log (Base 10): $ y = \log (x) $\\[14pt] % Line Break of 14pt (can use any unit of measurement)
    Log Base 2 (Binary): $y = \log_{2}(x) $\\[14pt]
    Log Base $e$ (Natural Log): $ y = \ln (x) $
\end{center}

\section{Roots:}
\begin{center}
    Square Roots: $ \sqrt{2} $\\[14pt]
    Cube Roots: $ \sqrt[3]{8} = 2 $\\[14pt]
    $n$th root: $\sqrt[n]{x}$\\[14pt]
    Pythagorean Theorem: $ r = \sqrt{x^2 + y^2} $\\[14pt]
    Square Root Inside a Square Root: $ \sqrt{1 + \sqrt{3x^2 + 3}} $
\end{center}

\section{Fractions:}

A Simple Fraction (Display Mode): $$ \frac{2}{3} $$

In a sentence (resized):

\begin{center}
    Is the glass $\frac{1}{2}$ empty or $\frac{1}{2}$ full?
\end{center}

In a sentence (Display Mode):
\begin{center}
    Is the glass $\displaystyle \frac{1}{2}$ empty or $\displaystyle \frac{1}{2}$ full?\\[14pt]
    With ams packages: $\dfrac{1}{2}$ empty or $\dfrac{1}{2}$ full?\\[14pt]
\end{center}

\pagebreak

More Complex Fractions: 
$$ \frac{\sqrt{x+1}}{\sqrt{x+2}} $$

$$ \frac{1}{1 + \alpha e^{-x}} $$

$$ \frac{x^3}{1 + \frac{1}{\sqrt{x}}} $$

\end{document}